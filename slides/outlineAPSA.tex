INTRO

Dynamic model seeks changes in ideal points.

Drift can be expected in Supreme Court: justices insulated from
political pressures + long tenure = opportunity to change opinions.

Drift NOT expected where agents are not insulated.

Judges (sincere?) vs. legislators (strategic?).


1. Patterns at IFE

Lots of movement, outliers. Two types: discrete jumps; monotonic
trends. Latter more problematic.

Description of what we did. Data, methods, dynamic more precise
estimation.

Spaghettis + E(median) (if zero harder to interpret, crossing median
seems big).

45� graphs (outliers) � not all move the same.


2. Potential explanations (representation/delegation)

A Freshman effect � adjustment. Partial renewal of IFE. Reference to
45�.

B Agenda control. B1 Self-restraint, perhaps election cycle loosens
agenda control further. Unavoidable conflict emerges more often +
party pressure. Polarization btw convergence within. # quejas.
Discrimination parameters. B2 Committee proposals should reflect
committee president's preferences; centrist vs extreme committee
votes?

C Sophisticated voting � affects center�s errors. Endogenous,
difficult to test. Partial replacement could change hopeless
alternatives to hopeful?

D Vote trading: affects center�s location. Endogenous,
difficult to test. Perhaps committee info could help.

E Changing milieu, exogenous. New Congress, party splits (59th Leg
PRI split).


3. Testing some propositions.



IFE has much less control of its agenda in electoral semesters. One
measure is the percentage of party complaints filed for the Council
General to resolve. Nearly two-and-a-half times higher in election
semester and immediate next than in non-election semesters.

--------------------------
 semester | mean(dparties)
----------+---------------
   1997s1 |       .1014493
   1997s2 **      .1935484
   1998s1 *             .1
   1998s2 |       .0694444
   1999s1 |       .1454545
   1999s2 |       .1034483
   2000s1 |       .2533333
   2000s2 **      .4175824
   2001s1 *       .2666667
   2001s2 |       .2531646
   2002s1 |       .2463768
   2002s2 |       .1147541
   2003s1 |       .1733333
   2003s2 **      .4644809
   2004s1 *       .2352941
   2004s2 |       .1410256
   2005s1 |       .1111111
   2005s2 |       .0827586
   2006s1 |       .2484472
   2006s2 **      .5436893
   2007s1 *       .3707865
   2007s1 |       .1413043
--------------------------


%%--------------------------
%%   delec1 | mean(dparties)
%%----------+---------------
%%        0 |       .1596467
%%        1 |       .3880597
%%--------------------------
%%. sum dparties
%%
%%    Variable |       Obs        Mean    Std. Dev.       Min        Max
%%-------------+--------------------------------------------------------
%%    dparties |      2410    .2485477    .4322607          0          1

\begin{tabular}{ll}
                            & percent party complaints \\ \hline
 Electoral semester + next  &  39 \\
 Rest                       &  16 \\
 All                        &  25 \\ \hline
\end{tabular}

We will need to join Omar's data and our own to analyze delta
(discrimination) parameter for party complaints vs rest; or even to
estimate ideal point on party complaints exclusively.





Within-party cohesion for election semesters and non-election
semesters, for PRI, PAN, PRD. Between-party polarization PRI-PAN,
PRI-PRD, PRD-PAN.

Delta (discrimination) parameters: \% not overlapping with zero for
election semesters and non-election semesters.

\% quejas for election semesters and non-election semesters
(Tema==2)

Buscar en base excel vieja columna de issue...

Preliminary comitee: financiamiento tema==3. Semster by semestre
share of divided votes with tema==3.



Appendix:

aCode

bGraphs with HPDs.
