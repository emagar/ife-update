
B Agenda control. B1 Self-restraint, perhaps election cycle loosens
agenda control further. Unavoidable conflict emerges more often +
party pressure. Polarization btw convergence within. \# quejas.
Discrimination parameters. B2 Committee proposals should reflect
committee president's preferences; centrist vs extreme committee
votes?

C Sophisticated voting � affects center�s errors. Endogenous,
difficult to test. Partial replacement could change hopeless
alternatives to hopeful?

D Vote trading: affects center�s location. Endogenous, difficult to
test. Perhaps committee info could help.

E Changing milieu, exogenous. New Congress, party splits (59th Leg
PRI split).


\begin{table}
 \begin{center}
  \begin{tabular}{lll}
  & From                     &  Prediction        \\ \hline
 1& Constituent pressure     &  Compare semesters with new Congress to rest.  \\
 2&                          &  Compare semesters following party split to rest.  \\
 3& Gatekeeping (& whipping?)&  Compare semesters with high \% party complaints to rest.  \\
 4& Sophisticated vote       &  Effect of partial replacement on continuing members. \\
 5& ---                      &  Freshman effect. \\
  \end{tabular}
  \caption{\emph{Testable propositions}}\label{T:testProps}
 \end{center}
\end{table}

3 Unlike modern legislatures, agenda control in IFE is quite
imperfect. Specialized committees operated in the Council in the
periods we scrutinize, certainly attempting to gatekeep undesirable
proposals off the plenary agenda. But their control is imperfect
because the parties whose very lives IFE regulates continuously
bring complaints and all sorts of issues to the Council's attention,
which the plenary cannot ignore. IFE, unlike the US Supreme Court,
has no docket control. La tercera fuente de conducta estrat�gica es
el presunto autocontrol de los consejeros como resguardo de
represalias desde su entorno pol�tico. En concordancia con la teor�a
de representaci�n que discutimos arriba, la presencia de actores con
poder para incidir, positiva o negativamente, en las carreras de los
consejeros es un fuerte aliciente para que �stos tomen en cuenta su
parecer. En nuestro trabajo previo sostuvimos que sobresalen cuando
menos dos actores en esta posici�n: las c�maras del Congreso, que
aprueban el presupuesto y pueden destituir a uno o m�s consejeros o
modificar la legislaci�n que rige la vida del IFE; y el TRIFE,
facultado para revertir o modificar los acuerdos y resoluciones del
IFE. La amenaza de un veto ex post debiera bastar para que los
consejeros moderen, ex ante, su conducta (McNollgast 1987). Habr�
decisiones que, por afectar a uno de estos actores externos, ser�n
prudentemente retiradas de la agenda. Esto resultar� en menor
polarizaci�n de la que, sinceramente, sostienen los consejeros.

4 La pregunta es si el voto sofisticado puede explicar
desplazamientos como los que detectan Martin y Quinn. Lo har�a si la
propensi�n al voto sofisticado (vinculada con las posibilidades de
triunfo de tu mejor alternativa) tuviese correlaci�n con alg�n
factor asociado al tiempo. Sin un cambio parcial en la composici�n
del Consejo General no vemos raz�n a priori para esperar esto. La
respuesta parece ser que el voto sofisticado no resultar� en
desplazamiento de puntos ideales con membres�a constante, pero puede
hacerlo tras un relevo parcial de consejeros. Hip�tesis 1 (de relevo
parcial): la salida de uno o m�s consejeros (pero no todos) en el
tiempo t y la subsecuente entrada de sus remplazos en t+1 pueden
producir un reacomodo de puntos ideales entre t y t+1 de los
consejeros moderados que permanecen en el IFE.

5 A freshman effect is plausible, recent appointees lack regulatory
experience and may show erratic behavior for a while. But if
predispositions for which agents were chosen are accurate, if
councilor sponsors in Congress selected types consistent with their
view of election regulation, then we would expect convergence of
members with the same sponsor quite rapidly after being appointed.
We discuss strategic considerations that can result in shifting
ideal points.
