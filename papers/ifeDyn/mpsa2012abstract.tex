Title: Partisanship Among the Experts: The Dynamic Party Watchdog Model of IFE, 1996-2012


Federico Estevez
ITAM
festevez@itam.mx

Eric Magar
ITAM
emagar@itam.mx

Guillermo Rosas
Washington Univ., St. Louis
grosas@wustl.edu

October 20, 2010

Abstract
We use a dynamic item response theory model to investigate ideological drift and stability in Mexico's IFE, charged with federal electoral regulation and composed of non-partisan experts selected by Congress. Results indicate that stability has predominated, but that several council members drifted over time to distinct positions, as revealed by their propensities to vote with or against other councilors. We highlight two sets of countervailing influences explaining such movements. One set, important for the relative stability of voting patterns, is the persistent bias introduced by partisan selection of council members, reflecting sponsors' long-term strategic imperatives in electoral regulation. Another set is related to IFE's institutional set-up, especially its committee system and the need for cooperation among councilors with divergent party sponsors. The gains from trade may be strong enough to offset partisan segmentation of the council, as was arguably the case in the first years examined, but not later. More generally, coalition patterns within IFE reflect remarkably well the wider political-electoral context of the last decades.

