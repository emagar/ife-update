\documentclass[12pt]{article}
\usepackage[latin1]{inputenc}
\usepackage{calc}
\usepackage{setspace}
\usepackage{graphicx}
\usepackage{multicol}
\usepackage[normalem]{ulem}
\usepackage[english]{babel}
\usepackage{color}
\usepackage{hyperref}
\usepackage[parfill]{parskip}
\usepackage[letterpaper,right=1in,left=1in,top=1in,bottom=1in]{geometry}

\begin{document}


\noindent \textbf{Ms. Ref. No.:  JELS-D-07-00004}

\noindent Partisanship in Non-Partisan Electoral Agencies and Democratic Compliance: Evidence from Mexico's Federal Electoral Institute
\bigskip

In responding to concerns raised by anonymous reviewers at
\emph{Electoral Studies}, we have revised our original manuscript
substantially.  Following editorial suggestions, we have removed
about one fifth of all footnotes and trimmed some others, as a
result of which the text is shorter but has gained in readability.
Thanks to detailed reviews of the previous version, we believe that
the current draft has also gained in scholarly significance.  Both
reviewers agreed that the ``party watchdog'' theoretical framework
within which we understand IFE was well articulated overall, but
both pushed us to clarify and point out the limits of our arguments.
We have endeavored to incorporate their criticisms and suggestions
in the revised version. We take this opportunity to thank the
reviewers, and to relate in detail the substantial changes in the
manuscript that have been prompted by their comments.

Both reviewers suggested that the original draft overstated our main
theoretical claim and exaggerated our ability to substantiate it. In
particular, Reviewer \#2 pointed out that we provided no evidence
that ``election arbiters that embrace partisan strife, rather than
those that expunge party politics...are better able to organize free
and fair elections'' (this point was echoed in comments by Reviewer
\#3).  Indeed, substantiating such a claim would require a different
research design, one that we cannot execute competently within the
limits of our paper.  We have toned down our over-reaching
statements wherever they appeared implicity or explicitly in the
paper, particularly in the abstract, introduction, and conclusion.

We do make an a priori claim that principals may find a partisan
checks-and-balances arrangement preferable to the ``ombudsman
model''.  We then show that the institutional setup of IFE
incorporates, to a large extent, partisan checks-and-balances.  Our
main purpose, though, is to furnish evidence that partisan strife
characterizes IFE.  This evidence comes predominantly from the
arrangement of ideal points that we have uncovered, which is
consistent with what we call the party sponsorship hypothesis (we
also refer to this, more circumspectly, as the party ``influence''
hypothesis).

Regarding this evidence, we acknowledge that the inferred ideal
points of Councilors are not always connected, as noted by Reviewer
\#3. But this is no impediment to reject the party influence
hypothesis in all three Councils. The reviewer's observations pushed
us to revisit nonparametric statistics manuals (Peter Sprent,
\emph{Applied Nonparametric Statistical Methods}, 1989, pp. 114-5
and Daniel 1990, now cited in the paper), leading us to realize that
the ANOVA one-way test we performed in the previous version was not
entirely appropriate given the nature of our data. Sprent (pp.
114-5) and Daniel (section 6.3) both agree that the ranks
Kruskal-Wallis test is preferable for data like ours: Kruskal-Wallis
is more robust than one-way ANOVA in the presence of non-normality,
heteroscedasticity, and outliers. Reasonable doubts can be raised
about the assumptions of homoscedasticity and no outliers implicit
in the ANOVA test.   ANOVA tests differences in means between
groups; Kruskal-Wallis differences in medians between groups. For
this reason, the latter is robust to the presence of outliers. The
Kruskal-Wallis test we perform in the revised manuscript now allows
us to reject the null hypothesis that councilors sponsored by PAN,
PRI, and PRD all come from the same population (Table 7).  Along
these lines, we have also added a straightforward exercise in basic
probability (p. 26):  We ask ourselves how likely it is that the
ideological positions of Councilors sponsored by the same party
would turn out to be adjacent to one another, given an assumption
that ideological positions are independent of party labels.   We
find that this event is quite unlikely under the assumption of
independence, with a probability that we estimate in about three in
a thousand.  Yet, we find that this event occurs in all three
councils, leading us to doubt the assumption that ideological
positions are not related to party sponsors.

As for the high proportion of consensual voting at IFE that Reviewer
\#3 notes, we do face a classic case of an equilibrium with
non-observable consequences.  This is because the electoral law that
gave rise to IFE in 1996 (still in force) is very detailed and
establishes many procedural constraints that the Council-General
must obey, as we relate in Section 2.  These constraints reflect the
three-party consensus over electoral regulation that existed in
1996.  IFE's Council General operates in the limited discretionary
range allowed by this law. The large amount of observed consensus
mostly corresponds to votes in areas where there is no space for
discretion (cf. McCubbins, Noll, and Weingast 1987).  Therefore, the
high proportion of consensual votes does not contradict the
hypothesis of partisanship, it merely complicates the analysis.

Reviewer \#3 raises some further issues about the statistical
analysis in Section 3.  First, we admit that our presentation of the
relevant data in the previous draft was uneven, given that we moved
from tacitly considering abstentions as dissenting votes in Section
2 to implicitly assuming they were missing at random in Section 3.
We have added prose that smooths over this inconsistency (for
example, referring to votes where at least one Councilor voted
against the majority or abstained as ``non-unanimous'', rather than
``contested''; see also fn. 11).  For the purpose of documenting
consensus in the Council-General we deem it appropriate to count
abstentions as votes that are not aligned with the majority, which
is why abstentions appear as one category of non-unanimous votes in
our aggregate description of the roll-call data.  \emph{But we have
no grounds to argue that abstentions always represent explicit
denouncements of the majority's position}.  Therefore, we are
satisfied with the MAR assumption that underlies our analysis, and
we now make it explicit that this is our building assumption.  In
other words, we assume that the pattern of missingness is random
conditional on item and location parameters; this assumption cannot
generally be verified, but it is a common one in the study of
roll-calls (Clinton et al. 2004).  Incidentally, one of the major
strengths of the Bayesian approach to item response models is that
it handles missing values not by dropping them from the analysis,
but by treating them as parameters to be estimated.  By imputing
missing values in the Gibbs sampling procedure, we appropriately
acknowledge the larger degree of uncertainty that we have about the
ideal points of Councilors who abstain disproportionately.  Such
added uncertainty would not be considered by methods that drop
missing values.

Reviewer \#3 raises another important theoretical issue, namely, the
sustainability of a party watchdog arrangement in view of the
temptation to shut one (or more) of the original principals out of
the electoral agency.  Indeed, this observation was prompted by the
decision by PRI and PAN to exclude PRD nominees from the 2003
Council-General.  We agree with the reviewer that our theory does
not specify the conditions under which we might expect exclusions in
what students of government formation may call the protocoalition
stage (Strom 1990).  We did not set out to do so originally, but
only (1) to suggest that successful electoral competition in a new
democracy can be (and has been) organized in such a way that parties
in the enacting coalition check one another in order to generate
trust and (2) that such a body behaves in a manner consistent with a
party watchdog setup.  In any case, the PRI-PAN decision to shut out
the PRD had more to do with this party's recalcitrant position
during the negotiations to name the 2003 Council-General than with a
preconceived plan by PRI and PAN to capture the Council.  We have
added a comment to this effect in the conclusion.

Regarding the parallels noted by Reviewer \#2 between our paper and
some themes raised in the literature on Judicial Politics in the US,
we have included relevant references and some discussion in Section
2.  First, we have added references to work on courts and their
influence in systems of separation of powers.  We do so in the
context of our discussion about ``non-statutory factors'' that may
limit the discretion of Councilors in Section 2.2.2 (p. 16). Second,
we provide citations to scholarly work that considers the
possibility of consensual opinions in lower courts as a strategy to
preempt revision by a higher court (fn. 12).  In the context of US
judicial politics, this theme appears most prominently in the
analysis of district courts decisions.  Since these decisions can be
overturned by the Supreme Court (an event that may cause
professional embarrassment), some claim that district courts may
have an incentive to seek consensus in order to diminish the
probability that the higher court may vote to overturn (Cameron et
al. 2000, Lax 2003, Songer et al. 1994).  This mechanism is
identical to the one we believe may explain the high degree of
consensual voting in Mexico's IFE.  However, where the literature on
US judicial politics concludes that incentives for strategic
consensus-building are low (the Supreme Court, after all, only
reviews a very small percentage of all cases decided by lower
courts), we believe that exploring this mechanism in the context of
Mexican politics would yield considerable theoretical payoff given
the relatively high proportion of decisions that are reviewed by
TRIFE and given that so many actors can start the review process.
This is indeed an opportunity for further research that we seek to
explore in another paper.

Finally, the reviewers also showed concern about the universality of
our claim, namely, that a party watchdog arrangement such as the one
we see in Mexico may be able to engender support.  Indeed, the
original draft was parochial in its inability to point to realms
outside of Mexico's electoral politics where this claim could be
explored.  To remedy this situation, we have rewritten the
conclusion to reflect our conjecture that party-watchdog schemes may
be superior forms of electoral organization and to point to other
(non-electoral) realms where agencies that embody strife have been
created to deal with lack of trust.

\end{document}
