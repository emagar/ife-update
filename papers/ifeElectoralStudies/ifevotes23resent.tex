\documentclass[12 pt, letter]{article}
\usepackage{amssymb, latexsym, amsfonts, amsmath}
\usepackage{setspace} %allows to change linespacing
\usepackage{url}
\usepackage[longnamesfirst, sort]{natbib}\bibpunct{(}{)}{,}{a}{}{,}
\usepackage{rotating}% allows sideways tables
\usepackage{color}
\usepackage{supertabular}
\usepackage[letterpaper,right=1in,left=1in,top=1in,bottom=1in]{geometry}
\usepackage{graphicx}
%\usepackage{epstopdf}
%\DeclareGraphicsRule{.tif}{png}{.png}{`convert #1 `dirname #1`/`basename #1 .tif`.png}
\hyphenation{re-pre-sent-ing}
\hyphenation{com-mit-tee}
\usepackage[nolists]{endfloat}  %Sends all figures, tables, to last pages
\usepackage{rotfloat}

\begin{document}

\begin{center}
\Large{Partisanship in Non-Partisan Electoral Agencies\\ and Democratic Compliance:\\ Evidence from Mexico's Federal Electoral Institute}
\end{center}
\bigskip

\begin{flushright}
``Democracy is a system in which parties lose elections.'' \\ \citet[p. 10]{Przeworski1991} 
\end{flushright}

\doublespace \noindent Przeworski's dictum implies that losers in a democracy should concede victory to winners. The reasons why losing parties choose to comply with the voters' verdict, however, remain a matter of debate. Some emphasize the key role of a civic culture shared by the citizenry \citep[e.g.,][]{Putnam2000}; others the need for democracy loyalists to stick together when extremists perform disruptive actions \citep[e.g.,][]{Linz1978}; still others underscore the importance of electoral rules and institutions in facilitating negotiation \citep[e.g.,][]{Jones1995}. As in \citeauthor{Przeworski1991}'s classic formulation, we emphasize the importance of trust in the fairness of the electoral process. Losers abide by the ballot box because they retain a chance of winning a fair election in the future.

Indeed, the scholarly literature on ``agencies of restraint'' portrays Electoral Management Body (EMB) independence as a solution to impart credibility to electoral processes routinely marred by fraud \citep{Schedler1999, Hartlyn2006}.  Scholars and practitioners suggest that technical experts that oversee a professional civil service staff, are secure in their tenure, and enjoy ample, non-political budgets are in an optimal position to guarantee fair electoral processes \citep[see references in][]{LopezPintor2000, Mozaffar2002}. Instead, we suggest that electoral credibility and compliance with electoral outcomes can also be achieved by granting political parties strong influence within EMBs, provided that all main parties see themselves represented in their deliberations.

To substantiate this interpretation, we study electoral regulation in Mexico during the last decade.  Mexico provides an interesting case-study into the organization of credible elections.  In a view prevalent among pundits and scholars, Mexican elections became credible once the Institutional Revolutionary Party (PRI), which had been losing votes for at least 20 years, finally relinquished control over electoral regulation to an autonomous EMB, the \emph{Instituto Federal Electoral} (IFE) in 1996. IFE's Council-General personified non-partisan expertise unencumbered by direct political interference from government. Councilors were thoroughly vetted and recruited from a set of professionals without party affiliation and admitted to the council after winning consensual endorsement in the Chamber of Deputies. Once in office, IFE's operational budget, which includes generous public financing for political parties and their electoral campaigns, is subject to few political whims.  Consistent with the view that EMB autonomy is paramount in guaranteeing fair elections, IFE is often heralded as exemplary of the ``ombudsman'' model of electoral management \citep{Eisenstadt2004}, which welcomes delegation of electoral authority to agencies staffed by detached, non-partisan experts.

IFE's Council-General ran three competitive elections in which incumbents suffered important losses. The PRI lost control of Congress after seven decades of uninterrupted rule in the 1997 midterm election, and then lost the presidency to the center-right National Action Party (PAN) in 2000. The PAN was relegated to second place in Congress in the 2003 midterm election. Yet, political losers accepted the outcomes of federal elections in these years.  But the new Council-General, inaugurated in 2003, lacked consensual support in the Chamber of Deputies.  At the last minute, the center-left Party of Democratic Revolution (PRD) broke off negotiations and was ignored by the other major parties during the process of  councilor selection.  The PRD immediately accused IFE of lacking autonomy from political parties, and then failed to concede victory in the 2006 presidential election.  

We do not contest the claim that IFE's Council-General brought credibility to elections during Mexico's protracted transition to democracy.  On the contrary, IFE remained one of the most respected democratic institutions in Mexico, at least until after the post-election dispute of 2006 temporarily dented its wide support.  However, we are not persuaded that this is the consequence of non-partisan impartiality of IFE's Councilors, as the ombudsman model assumes. Nor do we agree with the view that the post-electoral debacle of 2006 is the result of replacing an autonomous IFE with a particised one. Instead, we analyze IFE's institutional setup as a response to standard problems of delegation: parties carefully select their representatives to IFE and have tools to induce them, within limits, to act in accordance with their interests. Our view is that political parties, not non-partisan technocrats, have always been the ones that run the show at IFE.  In this regard, IFE should be considered to be closer to a checks-and-balances ``party watchdog'' \citep{Molina1999} model than to the ombudsman model of EMB organization.  And it was precisely the lack of full checks-and-balances---\emph{not} a rebirth of undue partisan influences---that made IFE's authority questionable.

To substantiate the view that parties have not relinquished control over the agent in charge of regulating their own behavior, we start in Section~\ref{S:delegation} by fleshing out the various dilemmas that politicians confront in delegating authority to a regulatory agency. We describe IFE's institutional design in Section~\ref{S:description}, showing how, despite central guarantees against majority tyranny, parties avail themselves of an array of resources to influence decisions in the Council-General. Ultimately, IFE's institutional setup suggests that councilors will be sensitive to the goals of their party sponsors, even in the absence of formal ties to political parties.  We refer to this as the \emph{party influence} hypothesis.  We then inspect councilor behavior for traces of partisanship in Section~\ref{S:estimation}. We employ Bayesian MCMC estimation techniques to examine the voting record of all Electoral Councilors between October 30, 1996, and August 24, 2005, spanning two entirely different councils.  Based on this analysis, we can make relatively precise inferences about the ideal points of council members in one-dimensional ideological space.  Our statistical analysis uncovers patterns consistent with the party sponsorship interpretation of councilors' voting behavior both before and after the 2003 renewal of the Council-General.

\section{Delegation Dilemmas}\label{S:delegation} We see IFE as an agent of political parties in the Mexican legislature, who have the power to appoint and remove the organ's governing board. Consequently, we analyze its institutional setup within the canonical literature on delegation.\footnote{For a general discussion of the logic of delegation, see \citet{Kiewiet1991}, pp. 22-38.}  In this light, political parties in the enacting coalition delegate authority to interpret the law and run federal elections to their appointees in IFE's Council-General.  Two critical delegation problems arise from the perspective of the enacting coalition.   The first problem is that appointees may behave in ways that do not serve the principals' common interests, i.e., delegation is open to moral hazard. The other problem arises from the fact that the enacting coalition is itself a collective principal, whose members have conflicting electoral interests. The enacting coalition would like to set ample \emph{agency discretion} to take advantage of agent specialization, while at the same time limiting the possibility of excessive \emph{agency loss} \citep{Madison1788, McCubbins1987, Huber2002, Miller2005}.  The literature on delegation dilemmas has uncovered various, and often ingenious, ways of mitigating this tradeoff.  But the heterogeneous nature of the enacting coalition complicates delegation by broadening the definition of agency loss.  In this sense, IFE is a regulatory board that must serve the interests of a very broad constituency---legislative political parties with opposing electoral interests and views on electoral law that are often incompatible---while striving to achieve credibility and trust in elections.

For political parties in the enacting coalition, the relevant question is how much influence on IFE can they retain while still achieving credibility in the eyes of the electorate.  We interpret recent work on ``agencies of restraint'' \citep{Schedler1999, Eisenstadt2004} as suggesting that anything short of full abdication (``autonomy'') impedes trust and credibility.  It is important to ask who is involved in the credibility game to evaluate this conclusion.  The tacit premise is that citizen's trust in IFE's independence and impartiality matters most.\footnote{Before the 2006 presidential election, public opinion decidedly backed IFE; for example, in a May 2005 survey nearly two-thirds of respondents considered IFE trustworthy, more so than any other political institution in the country.  The organizations that IFE regulates received much less support in citizen evaluations, as only one in three respondents expressed any degree of trust in political parties.  See national face-to-face survey, May 20-22, 2005, in \emph{Reforma}'s supplement \emph{Enfoque}, June 5, 2005, p. 6.} Instead, we start from the premise that trust in the fairness of the electoral process is most important to political parties, who are after all directly subject to IFE's regulatory and procedural decisions.  Unless political parties believe in IFE's aura of effectiveness and fairness, they will not be willing to concede defeat in the competition for power.\footnote{Citizen trust is important, but public opinion tends to follow party cues.  Panel surveys reported by \citet{Estrada2007} allow comparisons of democratic attitudes among supporters of losing candidates. In 2006, PRD voters were more distrustful, with as many as a quarter of them expressing, both before and after the election, that Mexico was not a democracy, compared to 13\% of PRI voters. After the election, another quarter of PRD voters, who had originally expressed  trust in democracy, joined the distrustful camp after losing; only 10\% of PRI voters did the same. Thus, after the election, 50\% of PRD voters disputed the democratic character of the Mexican political system, compared to slightly less than a quarter of PRI voters and 16\% of PAN voters.  These differentials are likely products of elite cues in the pre- and, especially, post-electoral disputes.}

Given our view that winning the trust of political elites is more important than generating confidence among the electorate, \emph{it is crucial that IFE remains subject to partisan influence}. We argue that, in structuring IFE, parties have managed to retain influence in the Council-General while at the same time checking each other's ambition.  The solution takes advantage of the agent's collective nature: the Council-General is a nine-member body sitting atop IFE's bureaucracy.  By ensuring that all relevant political parties have Council representation, the enacting coalition imposes a checks-and-balances solution to the aforementioned dual delegation problem. Minority representation, rather than an autonomous ombudsman, protects the interests of parties with IFE representation while simultaneously ensuring that no single party will be able to manipulate elections.

\singlespacing
\section{IFE's institutional design:\\  The party influence hypothesis}\label{S:description}

\doublespacing IFE was established in 1990 as a bureaucratic agency in charge of overseeing federal elections.  Although its original charter called for a preponderant presence of the Executive branch on its board, successive reforms led to the creation of a vigorous agency independent from Mexico's once omnipotent Presidents.  Concurrent with its increasing autonomy, IFE took over the years an expanding role in organizing all electoral aspects of Mexico's protracted transition to democracy.  Today, IFE's Council-General decides on all organizational matters relating to elections, including voter registration, district boundaries, operation of electoral booths, vote counts, monitoring of party and campaign expenditures, and overall regulation of political campaigns and party organization.

IFE took its present form during the last major election reform in 1996.  The size of the Council-General was set at nine members, eight of whom are non-partisan ``Electoral Councilors'' selected and ratified by consensus among congressional parties.  The Minister of the Interior---who chaired the council \emph{ex officio}---was removed from the council altogether and replaced by a non-partisan Council President chosen through the same consensual procedures.  In effect, the Executive relinquished day-to-day control over electoral matters and IFE became an autonomous regulatory agency.\footnote{\citet{Malo1996} analyze roll call votes from 1994-1995 in search of the determinants of individual vote choices.  They find that the six non-partisan Citizen Councilors, selected by the consensual procedures retained in the 1996 reform, tended to vote as a bloc, largely isolating the Legislative Councilors who directly represented the major congressional parties.} Scholars ultimately explain delegation to IFE technocrats as a constrained but purposeful move by PRI leaders to benefit from clean elections, given their calculus that the party would maintain sufficient citizen support to win them \citep{Magaloni2006}.  While that calculus proved wrong, the battle for credibility was clearly won. However, the influence of congressional parties over the council's composition leaves ample room for speculation about potential party sponsor effects on the voting behavior of councilors.  In order to orient our investigation of that behavior after the 1996 reform, we turn to a detailed discussion of IFE's institutional design, underscoring those rules that provide incentives for \emph{pro-sponsor behavior}, in contrast to those that induce \emph{cross-partisan voting} or even outright universalism.

\subsection{Incentives for partisan voting behavior}
Following principal-agent theory, we emphasize three aspects that are relevant in generating pro-sponsor or partisan voting behavior: rules of nomination, signaling devices used by sponsors, and party capture.

\subsubsection{Rules of nomination}
Councilors are appointed by a two-thirds vote in the Chamber of Deputies to serve seven-year terms.  Tenure in office is fairly secure, yet Congress can impeach any councilor---a possibility we discuss at length below.  Legislative parties have informally agreed, in bargaining sessions over councilor selection since 1994, that no single party should designate a majority on the council, that each party in the enacting coalition is entitled to propose a share of councilors roughly proportional to its lower chamber seat share, and that nominated candidates can be vetoed by any other party in the coalition \citep{Alcocer1995, Schedler2000a}. After the election of a single nominee for Council President, a final logroll in the lower chamber on a closed list of eight candidates (plus a ranked list of replacements) culminates the process.  In 1996, all parties with congressional representation (PRI, PAN, PRD, and PT) joined the enacting coalition; in 2003, only three of six congressional parties were included.\footnote{The PT is the \emph{Partido del Trabajo}.  In 2003, the PRD and PT were excluded from the enacting coalition, while the \emph{Partido Verde Ecologista Mexicano} (PVEM) was incorporated.}  Table \ref{T:proposals} shows information about the enacting coalitions formed in 1996 and 2003, the relative strength of legislative parties, and the number of candidates that each party successfully sponsored to the Council-General.

\begin{sidewaystable}
\caption{Legislative party shares, enacting coalitions, and
councilor sponsorship}\label{T:proposals}
\begin{center}
\begin{tabular}{lccccccc}
\hline\\ [-1.5ex]
      & $56^{\text{th}}$ Leg. & Woldenberg I & $57^{\text{th}}$ Leg. & $58^{\text{th}}$ Leg. & Woldenberg II & $59^{\text{th}}$ Leg. & Ugalde \\
Party &  `94-`97  &  `96-`00  &  `97-`00  &  `00-`03  &  `00-`03*  &  `03-`06  &  `03-`10 \\
\hline\\ [-1ex]
 PAN  & \textbf{24\%} & 2   & 24\% & 41\% & 2   & \textbf{30\%} & 4   \\
 PRD  & \textbf{13\%} & 3   & 25\% & 10\% & 2   & 19\%          & --- \\
 PRI  & \textbf{60\%} & 3   & 47\% & 42\% & 4   & \textbf{45\%} & 4   \\
 PT   & \textbf{ 2\%} & 1   &  1\% &  1\% & 1   & 1\%           & --- \\
 PVEM & ---           & --- &  1\% &  3\% & --- & \textbf{3\%}  & 1   \\
 Others & ---         & --- &  --- &  1\% & --- & 1\%           & --- \\
 N    & 500           & 9   &  500 &  500 & 9   & 500           & 9   \\
\hline
\multicolumn{7}{l}{\footnotesize{*Two councilors resigned in late 2000 and were replaced by substitutes pre-selected in 1996.}}\\
\multicolumn{7}{l}{\footnotesize{Enacting coalition in boldface.}}\\
\end{tabular}
\end{center}
\end{sidewaystable}

Party sponsorship quotas in the Council-General have always been flexible, overrepresenting the PRD until 2003.  In 1994, the PRD received three of six Citizen Councilors.  The PRI accepted this distribution shielded behind its control of the then powerful Council presidency.  The PRD again demanded a large representation in 1996 and got it, although its fourth member had to be co-sponsored with the PT (Councilor Cant\'u).  The PAN accepted to sponsor only two members in exchange for strengthening the Council-General's powers vis-\`{a}-vis IFE's bureaucracy.  In 2003 it was the PRI, having just re-gained plurality status in the chamber, that demanded four seats in the new Council-General.  

While the informal right to veto eliminates highly partisan and otherwise unqualified candidates, it is unlikely that any party would nominate individuals clearly opposed to its own interests and views about electoral regulation.\footnote{\citet{Schedler2000a} expresses a similar view about the selection of council members, but considers that their conduct once in office must demonstrate prudence and impartiality in order to accomplish their task.  That they should appear to be purer than Caesar's wife, however, does not necessarily make them so.} Parties reduce the chances of selecting ``bad types''---i.e., individuals whose conduct could hurt the sponsor's interests---by screening potential agents carefully and proposing candidates who, while unaffiliated to them, have preferences in line with those of the chamber party.  Thus, screening helps mitigate agency costs.  As in \citeauthor*{Cox1993}'s \citeyearpar{Cox1993} congressional committees, the resulting Council-General can be seen as a microcosm of the enacting coalition in the lower chamber, with councilors keeping tabs on each other by defending their sponsors' interests in IFE's debates and decisions.

Contrary to what had happened in previous negotiations, no major party was willing to concede to the PRD's maximalist demands in 2003, leaving it without representation in IFE's Council.  This sign was ominous from the partisan perspective we develop, a harbinger of electoral distrust to come arising from incomplete checks-and-balances in the renewed Council-General.

\subsubsection{Signaling devices used by sponsors}
Should councilors shirk or deviate from their sponsors' expectations about appropriate voting behavior, parties retain a wide gamut of mechanisms to make their preferences known to agents and call them to order.  The mechanisms include positioning in council and committee debates,\footnote{The 1996 reform introduced committees for each of IFE's operational areas, staffed through voluntary participation of individual councilors, and with chairs assigned by consensus in the council. Party organizations with legal registry exercise voice without a vote on the Council-General and in committee.} public and private communications of all sorts (including threats of impeachment against council members), agenda interference through the filing of petitions and complaints, and recourse to appeal before an electoral tribunal (we expand on some of these below).  These mechanisms should make sponsor preferences completely transparent to councilors.

\subsubsection{Party capture}
Assuming councilors are ambitious and have reasonably low discount rates for the future, their expectations of post-IFE careers may be molded by offers of continued sponsorship (or, indeed, by offers from rival parties).  The danger of ``party capture'' was present from the outset, but the original legislation and its reforms in the 1990s ignored the problem.\footnote{An initiative to restrict employment of ex-Councilors in government positions or elective office has been frozen in Congress since 2001.}  Table \ref{T:postife}---which includes the list of Citizen Councilors from 1994 to 1996---speaks to this issue.  Ironically, the parties that most demanded electoral impartiality and citizen control have tended to advance the post-IFE careers of their nominees, while the former ruling party has largely abandoned its own.  In any event, a party can offer future-oriented incentives to its nominees in the hope of eliciting appropriate voting behavior.  Alternatively, parties can exploit the individual expectations of council members that professional opportunities may materialize in the future.

\begin{table}
\caption{Post-IFE Careers of Electoral Councilors}\label{T:postife}
\begin{center}
\begin{tabular}{llp{3in}}
\hline
Councilor & Sponsor & Post-IFE career \\ \hline \\ [-1.5ex]
\multicolumn{3}{l}{\underline{Carpizo Council (1994-1996)}}\\  [1.2ex]
Creel       & PAN & PAN Deputy (1997-2000), PAN candidate for Federal District Gov't (2000), Minister of the Interior (2000-2005), PAN Senator (2006-2012). \\ [0.5ex]
Woldenberg  & PAN & PRI nominee for Council President (1996). \\ [0.5ex]
Granados    & PRD & PRD gubernatorial candidate in Hidalgo (1998). \\ [0.5ex]
Ortiz       & PRD & PRD Deputy(1997-2000 and 2003-2006), PRD cabinet member in Mexico City Gov't (2001-2003). \\ [0.5ex]
Zertuche    & PRD & PRD nominee as IFE's Secretary-General (1999-2003). \\ [0.5ex]
Pozas       & PRI & Returned to academic life. \\ [1.2ex]
\multicolumn{3}{l}{\underline{Woldenberg Council (1996-2003)}}\\ [1.2ex]
Barrag\'an  & PRD & Returned to academic life. \\ [0.5ex]
C\'ardenas  & PRD & Returned to academic life. \\ [0.5ex]
Zebad\'ua   & PRD & PRD Secretary of the Interior in Chiapas (2000-2003), PRD Deputy (2003-2006), minor-party candidate for governor of Chiapas who withdrew along with PAN's candidate (2006). \\ [0.5ex]
Cant\'u     & PT  & PRD nominee (vetoed) for Council President (2003). \\ [0.5ex]
Lujambio    & PAN & PAN appointee as IFAI Commissioner (2005-2012). \\ [0.5ex]
Luken       & PAN & Returned to private business. \\ [0.5ex]
Molinar     & PAN & PAN Under-Secretary of the Interior (2000-2002), PAN Deputy (2003-2006), Director of Social Security Institute (2006-). \\ [0.5ex]
Merino      & PRI & Returned to academic life. \\ [0.5ex]
Peschard    & PRI & Returned to academic life. \\ [0.5ex]
Rivera      & PRI & Returned to academic life. \\ [0.5ex]
Woldenberg  & PRI & Returned to academic life. \\ \hline
\end{tabular}
\end{center}
\end{table}

\subsubsection{Expected partisan behavior}
The rules and devices outlined above lead us to expect that council members will represent their sponsoring party's views on electoral regulation, i.e., \emph{councilors should manifest partisan behavior}.  But it is also true that the broad lines of much of the Council-General's day-to-day business are inscribed in election statutes which have seen few significant changes since 1996 and which contain norms that reflect the principals' shared interests in electoral regulation.  From this perspective, the Council-General can be said to operate on \emph{autopilot}, executing standing agreements among the members of the 1996 enacting coalition.  In consequence, a large volume of decisions should be characterized by consensus among council members.  In addition, councilors retain substantial control over IFE's agenda and conceivably use it to prevent extremely divisive items from entering debates and votes in the Council-General.

Thus, open conflict in the Council-General should only occur sporadically, and only regarding topics that escape the gate-keeping control exercised by the Council. We have detected three types of topics with such characteristics: (i) internal agency matters, such as administrative appointments and budgetary decisions; (ii) electoral issues brought by actors outside the enacting coalition, which must be processed by IFE under threat of judicial reprimand; and (iii) issues whose emergence and divisive potential could not have been foreseen by the principals when they designated council members.

A preliminary inspection of roll call votes at the Council-General confirms the presence of strong consensual or cross-partisan tendencies.  The general lack of conflict among councilors can be ascertained from Figure \ref{F:unan}.  Vertical lines indicate changes in council membership, the first marking the exit of councilors Molinar and Zebad\'ua---who assumed government appointments in 2000 and were replaced by Councilors Luken and Rivera---the second marking the beginning of a completely renovated Council-General in November 2003.  Throughout the article we label these Councils-General by the names of their respective presidents: Woldenberg I (1996-2000), Woldenberg II (2000-2003), and Ugalde (2003-2005).  The top line in Figure \ref{F:unan} counts all roll-call votes observed each semester in the period analyzed.  The volume of IFE decisions is substantial---1,699 votes are included in the dataset---and peaks, as one would expect, in federal election years.  The middle line represents the number of \emph{non-unanimous votes}, i.e., those in which at least one councilor voted differently from the others or abstained, for a total of 728.  Unanimous votes above that middle line comprise 57\% of all roll-calls. The lower line in Figure \ref{F:unan} follows from a slightly stricter definition of conflict.  It registers all non-unanimous votes in which at least two councilors voted against the majority, excluding abstentions.  On this still modest definition of conflict, less than 16\% of all roll-calls at IFE would qualify as divided votes in the period under scrutiny, although its incidence has grown steadily from 1-in-10 under Woldenberg I, to 1-in-5 under Ugalde.

\begin{figure}
\begin{center}
   \caption{Unanimous, contested, and minimally conflictive Council-General votes, 1996-2006}\label{F:unan}
   \includegraphics[width=120mm]{"figure_1"}
\end{center}
\end{figure}

The high degree of universalism in the Council-General certainly deserves future attention, but one cannot infer that ``ideological reasoning'' is exceptional in IFE from the prevalence of consensual votes. After all, if the enacting coalition could anticipate all future conflicts in electoral regulation and if the Council-General had perfect control over its agenda, statutes would internalize all conflict and all decisions would possibly be reached by consensus---the autopilot analogy. Our research takes advantage of real limitations both in foresight and in the council's agenda control, which allow latent conflict to become observable. \emph{We expect that this conflict, however low its incidence, will nonetheless expose the ideological divergence and partisan predispositions of council members}.  When conflict arises, behavior by any councilor should dovetail her sponsor's interests and preferences. We therefore entertain the expectation that same-sponsor nominees will exhibit similar behavioral patterns on the council. From the perspective of political parties, deviations from this expected behavior can be seen as agency loss. Even allowing for slack due to vote-trading and idiosyncratic behavior, we still expect to find that same-sponsor councilors are ideologically closer to each other than to colleagues sponsored by different parties.  We test this party influence hypothesis in Section~\ref{S:estimation}, where we examine roll-call behavior in the Council-General.  Before doing so, we discuss other institutional design features that play \emph{against} our chances of detecting partisan behavior at IFE.

\subsection{Incentives for consensual behavior}
The consensual tendencies discussed so far are the product of \emph{ex ante} agreement among congressional parties in the enacting coalition.  Further inspection of IFE's institutional design reveals additional incentives of an \emph{ex post} nature for councilors to form cross-partisan coalitions when voting.  Here, we refer to two such incentives: the threat of impeachment and the existence of an electoral tribunal of last resort.

\subsubsection{Rules of impeachment}
Although the foremost objective of the 1996 reform was to grant autonomy to the Council-General, the delegation contract retains one important element to constrain agency behavior: the threat of impeachment \citep{Eisenstadt2004}.  A simple majority vote in the lower chamber is needed to indict a councilor, although a two-thirds vote in the Senate is required for actual impeachment.  In principle, a coalition of any two of the three largest parties could have sustained a majority vote in the Chamber of Deputies against any councilor at any moment since the PRI lost its congressional majority in 1997.  However meager the likelihood of destitution by the Senate, merely initiating the trial in the lower chamber might well suffice to destroy the career of any councilor.  No Electoral Councilor has yet undergone an impeachment trial, although the so-called ``Councilor Magistrates'' elected to eight-year terms in 1990 were summarily dismissed upon the approval of the election reform of 1994, thereby setting an ominous precedent against security of tenure at IFE.\footnote{Though no Councilor has ever been indicted, formal complaints have been filed multiple times and threats of impeachment invariably characterized by charges of overt partisanship are quite common. A search of newspaper \emph{Reforma}'s database since 1996 uncovered a total of 41 impeachment threats articulated by political parties (28 were issued during Woldenberg I, eight during Woldenberg II, and five so far under Ugalde).  Four ``official complaints'' (a prelude to impeachment) were jointly filed in 1999 by the PRI, PT, and PVEM.  There are also reports of four bills of impeachment sent to Congress in 2002, but these were mooted and left no trace in the record.  Of the twenty individuals occupying councilor positions since 1996, thirteen received threats of impeachment.} 

Under these circumstances, even ideologically-motivated councilors would shirk to some degree in order to protect their flanks against accusations of flagrant partisanship.  In order to secure their tenure, councilors should strive to act in ways that do not systematically hurt the interests of parties with combined majority support in the lower chamber.  This can be achieved by sometimes failing to toe the party line, and accommodating instead the interests of other parties and their council nominees.  Table~\ref{T:unidiv} categorizes roll-call votes in IFE's Council-General by the degree of unity manifested by party contingents of Electoral Councilors.  For example, the PAN successfully sponsored two councilors to the Woldenberg I Council.  In contested votes in which both were present, the pair voted in the same direction in 206 instances, while in 26 votes they parted company.  For purposes of the analysis presented in Table~\ref{T:unidiv}, we count an abstention as a dissenting vote, which opens the rare possibility that each Councilor will vote differently in a three-member contingent (this happened eight times in the PRD contingent in Woldenberg I, six times in the PRI in Woldenberg II).\footnote{In Section~\ref{S:estimation}, we relax the assumption that abstentions are dissenting votes and model them more appropriately as data that are missing at random.} All multi-member party contingents have shown some level of division in roll call votes, but there is wide variation across parties (with the PRD blocs by far the least unified) and across Councils (the current Ugalde Council shows a strong surge in disunity for PAN and PRI blocs). Shirking of this sort is very likely a matter of sincere preference revelation by individual councilors in many cases.  But whatever the motivation, deviation from the party line can often signify alignment with other partisan contingents on the issue at stake.

\begin{table}
\caption{Unity and division in multi-member party contingents (contested votes with no absent contingent members)}\label{T:unidiv}
\begin{center}
\begin{tabular}{lccccccccc}
\hline\\ [-1.5ex]
Sponsor & Dissenting & \multicolumn{2}{c}{ Woldenberg I} & & \multicolumn{2}{c}{ Woldenberg II} && \multicolumn{2}{c}{Ugalde} \\
 & Votes in & \multicolumn{2}{c}{1996*-2000} & & \multicolumn{2}{c}{2000-2003} && \multicolumn{2}{c}{2003-2006*} \\ \cline{3-4} \cline{6-7} \cline{9-10}
 & Contingent & Freq. & Pct. && Freq. & Pct. && Freq. & Pct.  \\
\hline \\ [-1ex]
PAN & 0 & 206 & 89\% & &261 & 83\% && 41 & 28\% \\
 &    1 &  26 & 11\% & & 52 & 17\% && 80 & 55\% \\
 &    2 & --- & ---  & &--- & ---  && 25 & 17\% \\ [1.3ex]
PRI & 0 & 218 & 90\% & &279 & 87\% && 51 & 35\% \\
 &    1 &  23 &  9\% & & 34 & 11\% && 67 & 47\% \\
 &    2 &   2 &  1\% & &  6 &  2\% && 26 & 18\% \\ [1.3ex]
PRD & 0 &  18 &  8\% & & 90 & 27\% && --- & --- \\
 &    1 & 212 & 89\% & &243 & 73\% && --- & --- \\
 &    2 &   8 &  3\% & &--- & ---  && --- & --- \\
\hline \multicolumn{10}{l}{\small *The series of roll-call votes for the Woldenberg Council starts October 1996;}\\
\multicolumn{10}{l}{\small the series for the Ugalde Council is truncated in June 2006.}
\end{tabular}
\end{center}
\end{table}


\subsubsection{Vetoes by a court of last resort}
Most discussions of IFE's institutional incentives tend to overlook the impact of a second actor, the \emph{Tribunal Electoral del Poder Judicial de la Federaci\'on} ({\sc Trife}).  Any Council-General decision can be appealed to this electoral court of last resort. All political parties and their candidates, national political associations, and even ordinary citizens in some cases, have standing before {\sc Trife} to challenge IFE's decisions.  Indeed, the tribunal has over the course of its history shown a growing interest in revising IFE's agreements, sometimes rewriting the tribunal's own jurisprudence in order to force its criteria on IFE, and at other times limiting the scope of IFE's decision-making power.  In many areas of election law, {\sc Trife}'s rulings have become unpredictable, and IFE decisions on the docket face rising odds of being overturned or amended.  Moreover, this behavior by the court has spawned litigiousness by those with standing to appeal \citep{Eisenstadt2004}.

{\sc Trife} is a busy court, as evidence in Table \ref{T:rulings} suggests, and has received a growing number of appeals since 1996.  Of the total of 1,699 roll-call decisions from the council, 218 have been challenged in court, involving 234 separate measures in 423 separate suits (IFE logrolls and multiple plaintiffs increase the number of appeals).  Moreover, the tempo of appeals has risen over time, from 1-in-9 decisions challenged during Woldenberg I, to 1-in-5 for the Ugalde Council.  At the other end, {\sc Trife} also grants appeals, in part or in whole, at twice its earlier rate, currently overruling IFE in one out of twelve roll-call votes.

Clearly, {\sc Trife} can be considered a ``nonstatutory factor'' that limits the discretion of IFE's Council-General \citep{Huber2002}. In some principal-agent accounts of delegation, such exogenous factors can assuage a principal's fear about potentially adverse agent behavior.  In this case, the ability of parties to challenge unfavorable council decisions \emph{ex post} should make them more willing to delegate power \emph{ex ante}, much as powerful political actors use the Supreme Court in the US to further their interests \citep{Clayton2002, Mcnollgast1999}. More importantly for our purposes, nonstatutory factors can also be expected to alter the behavior of agents.  In IFE's case, councilors who care intensely about some resolution have to anticipate all major complaints and make a priori concessions to preempt legal appeals from affected parties \citep[cf.][]{GelySpiller1990}. This can be achieved in two ways. First, councilors can craft resolutions that incorporate the tribunal's preferences based on precedent, hoping to avoid negative rulings in case of legal challenge. Second, councilors can reduce the probability that other actors, most prominently parties themselves, will appeal a decision. This route calls for compromise and accommodation in council decisions.\footnote{Indeed, Councilor Merino admits that consensual Council resolutions are more likely to withstand the scrutiny of {\sc Trife}  \citep{Merino1999}.  There is a similar argument in the literature on judicial politics in the US, according to which judges in District Courts may have an incentive to vote consensually so as to diminish the probability of review and a potential adverse ruling by the Supreme Court \citep{Cameron2000, Lax2003, Songer1994}.  As far as we can tell, this literature concludes that incentives for ``strategic consensus-building'' are low, if they exist at all, given the very small probability that the Supreme Court will review any one decision by a lower court.  Since {\sc Trife} reviews a much larger proportion of IFE decisions, strategic consensus-building may be more likely in the Mexican context.} An obvious implication is that council members will tend to form oversized, cross-partisan, and even universal voting coalitions. The obvious strategy for the councilors, given active engagement by the tribunal and increasing recourse to legal challenge, is to circle their wagons---that is, to seek safety in broad cross-partisan consensus.

\begin{table}
\caption{Legal appeals and {\sc Trife} rulings on IFE decisions, 1996-2006}\label{T:rulings}
\begin{center}
\begin{tabular}{llcc}
\hline\\ [-1.5ex]
Council & {\sc Trife} ruling &  N  & Pct. \\
\hline \\ [-1ex]
Woldenberg I & No appeal  & 572 &  89\% \\
1996-2000 & Appeal denied &  46 &   7\% \\
          & IFE overruled &  28 &   4\% \\
          & All           & 646 & 100\% \\ [1.2ex]
Woldenberg II & No appeal & 440 &  81\% \\
2000-2003 & Appeal denied &  60 &  11\% \\
          & IFE overruled &  40 &   7\% \\
          & All           & 540 & 100\% \\ [1.2ex]
Ugalde    & No appeal     & 171 &  79\% \\
2003-2005* & Appeal denied &  27 &  13\% \\
          & IFE overruled &  17 &   8\% \\
          & All           & 215 & 100\% \\
\hline \multicolumn{4}{l}{\small * Appeals to Ugalde's Council
tracked until April 2005.}
\end{tabular}
\end{center}
\end{table}


\subsection{The null and the party influence hypotheses}
When translated to the realm of Council-General roll-call votes, the null hypothesis of councilor autonomy states that knowing which party sponsored a given councilor \emph{should not} help us predict his or her voting behavior. The null follows from the perspective that the influence that sponsors may have is either non-existent, or is relatively tiny with respect to a multiplicity of other factors that systematically determine councilor votes in the Council-General. It is unclear which  factors are the ones that are most influential in allowing  councilors to remain ``autonomous'', but the prevalent view is that party labels clearly should not matter.

In contrast, our perspective is that incentives for pro-sponsor party behavior---which we find in nomination procedures, open signaling, and future rewards---are in fact the dominant factor.  As a consequence, knowing which party sponsored any councilor should be a strong predictor of voting behavior at IFE. We expect councilors' ideological positions to be distributed such that same-sponsor members occupy adjacent positions on the ideological space. In the extreme, the party influence hypothesis leads us to expect council members to cluster together in distinct same-sponsor blocs, thus defining a veritable partisan cleavage in the council.

Finding such clustering will not allow us to distinguish whether parties are finding good agents \emph{ex ante} through nomination procedures or whether they are eliciting pro-sponsor behavior \emph{ex post} through threats of punishment and/or signaling devices.  For our purpose, however, showing that the ideal points of electoral councilors are consistent with the ideological location of the parties that sponsored them suffices to show party influence. We test the party influence hypothesis through estimation of ideal points of IFE's Electoral Councilors during the period 1996-2005.

\singlespacing

\section{Ideology and partisanship in the Council-General}\label{S:estimation}

\doublespacing Political methodologists have developed various techniques to circumvent the ``micro committee problem'', i.e., the difficulty of estimating parameters of interest when the number of committee members is small, even if the committee has produced a long list of contested votes \citep{Londregan2000}.  Among these techniques, Bayesian methods \citep{Martin2002, Clinton2004, Jackman2001} are more appropriate to the study of individual voting behavior in small committees than other tools of ideal point estimation.  Since IFE's Council-General is a very small decision-making body---and, to further complicate matters, a highly consensual one---Bayesian Markov chain Monte Carlo item-response theory  offers the best way to generate valid inferences about councilors' ideological profiles, provided that our models are appropriately identified through suitable priors.

We present an analysis of IFE's two Councils-General in the period 1996-2005, but we break up the Woldenberg Council into two separate entities, as discussed in Section~\ref{S:description}.  We estimate ideal points for twenty individuals (seven of whom served throughout the Woldenberg years, so the ideal points of these individuals are estimated twice).  Our decision to study these councils separately stems from our interest in understanding whether individual ideal points stack in ways consistent with the party influence hypothesis, even after some councilors leave and others replace them. The large number of unanimous votes (971 in total) convey no information about councilors' ideologies and have been dropped from the analysis.  The remaining 728 usable votes are coded so that, in each case, an Aye vote is coded ``1'' and a Nay vote ``0''.  In Section~\ref{S:description}, we characterized votes with abstentions and absences as non-unanimous votes, implicitly admitting that abstainers were voting against the majority.  However, we have no grounds to believe that abstentions should be treated as Nay votes.  We thus prefer to see abstentions as missing parameters to be updated based on information from observed votes and conditional on values of individual- and item-specific parameters.  We do assume, as is common in ideal point estimation, that abstentions are ``missing at random'' \citep{Little1987, Clinton2004}.  Thus, our estimates of councilors' ideal points appropriately incorporate uncertainty generated from abstentions.

%For a discussion of alternative ways of modeling abstentions in IFE's Council-General see  \citet{Rosas2005}.}

We base our inferences on \citeauthor*{Clinton2004}' voting behavior model \citep{Clinton2004, Martin2002}.  The identification of this model requires imposing restrictions either on item parameters or on voters' positions.  Traditionally, scholars use a known ``extremist'' in the committee to anchor the ideological space, thus solving the problem of rotational invariance.  We use the alternative method of restricting the discrimination parameters of two items (i.e., two specific roll-calls) per council.\footnote{We stipulate standard normal prior distributions on councilors' ideal points to solve the problem of scaling invariance. For each of our datasets, we ran 200,000 iterations of the WinBugs sampler, discarding 100,000 as burn-in and thinning the resulting chain so as to keep 10,000 draws from the posterior distribution for inference purposes. We monitored convergence through Geweke's statistic.  Samples and convergence results are available from the authors for inspection.} In every case, we chose votes with substantive contents that pit political liberals against political conservatives, thereby imposing some structure on the ideological space underlying the individual voting records for each period.  

\begin{table}
\caption{Votes used to anchor policy space for each Council}\label{T:priors}
\begin{center}
\begin{tabular}{lp{1.5in}p{2.2in}}
\hline
Date (vote number)   & Minority vote & Substance \\ \hline   \\ [-1ex]
\multicolumn{3}{l}{\underline{Woldenberg I}} \\ [1ex]
12/16/1997 (vote 28) & PRI, Barrag\'an (Nay)  & Agenda power for President (PRI-sponsored): Should Council-General ratify President's appointee for one administrative office? \\ [1ex]
11/14/2000 (vote 228)  & PRI, Barrag\'an (Aye) & Scope of IFE authority: Should PAN be held responsible and fined for the case of a clergyman who campaigned illegally on its behalf? \\ [1ex]
\multicolumn{3}{l}{\underline{Woldenberg II}} \\ [1ex]
4/6/2001 (vote 27) & C\'ardenas, Cant\'u, Luken (Nay) & Money in elections: Should IFE drop investigation of complaint by Alianza C\'ivica against the PRI for clientelistic practices in Chiapas? \\ [1ex]
5/30/2003 (vote 206)   & PRI (Aye) & Scope of IFE authority: Should PAN be fined for a TV campaign spot that PRI considers libelous? \\ [1ex]
\multicolumn{3}{l}{\underline{Ugalde}} \\ [1ex]
8/23/2004 (vote 33) &  PAN minus Morales, Latap\'i (Nay) &  Agenda power for President (PRI-sponsored): Should candidate for top-level appointment, proposed by Council President without relevant commission's consent, be ratified? \\ [1ex]
1/31/2005 (vote 43) &  Andrade, L\'opez Flores, Morales, G\'omez Alc\'antar (Nay) & Scope of IFE authority: Must PVEM statutes make party leaders accountable to rank-and-file?\\[0.5ex]\\ \hline
\end{tabular}
\end{center}
\end{table}

A discussion of this structure is in order. Most issues voted on at IFE fall into two general categories. The first category comprises decisions about the pace of reform to liberalize political competition. This category corresponds to the second dimension of Mexican politics described by \citet{Molinar1991} and \citet{Lujambio2001}, among others, and substantiated empirically by \citet{Moreno2003a} and \citet{Magaloni2006}.  Two examples of relevant items comprised by this category are: How easy should it be to replace top- and mid-level IFE bureaucrats appointed by the PRI before the electoral reform? Should IFE have full authority to penalize vote-buying?  The second category includes decisions on IFE's scope of authority to defend citizen rights against party encroachments \citep[cf.][]{Cardenas2004}: Should IFE intervene to ban negative campaigns ads, seen by many in Mexico as contrary to citizen interest? Or to defend the rights of rank-and-file party members against party leaders? Interviews with former Councilors Lujambio and Molinar corroborated that our choice of anchors, listed in Table~\ref{T:priors}, corresponds  to their perception of the principal themes discussed during their tenure at IFE and afterwards.  According to our interviewees, issues in the first category were dominant in the first years, issues in the second in the latter years, especially in the Ugalde Council. Yet, party positions on these  categories should overlap substantially, with the PRD and PRI standing on opposite ends of the spectrum, and the PAN somewhere in between, at times voting with the PRD, at others obtaining key concessions from the PRI in exchange for policy support in the Chamber of Deputies.

Table~\ref{T:idealpoints} reports councilors' ideal point estimates.  The last column in the table displays the number of votes on which we base our estimation of each councilor's ideology. Note that point estimates of ideal positions (the mean of the posterior distribution of the $9 \times 3$ location parameters) determine individual ranks and relative ideological distances within each council.  For example, the nine Electoral Councilors that served from 1996 to 2000 are aligned from left to right as follows: C\'ardenas, Cant\'u, Zebad\'ua, Lujambio, Molinar, Merino, Woldenberg, Peschard, and Barrag\'an.

The distribution of ideal points in the Woldenberg I Council is largely supportive of the party influence hypothesis, showing tightly adjacent positions for both the two PAN nominees and the three PRI nominees.  The glaring anomaly is Barrag\'an's position at the extreme right of the ideological spectrum, when other members of the PRD contingent (and the sole councilor sponsored by a smaller left-wing party) clearly occupy the left end of the scale.  This outlier would appear to be an example of deficient screening by his party sponsor, a singular exception to partisan segmentation of the Council.

\begin{table}
\caption{Posterior distribution of ideal points}\label{T:idealpoints}
\begin{center}
\begin{tabular}{llrrr}
\hline
 Councilor   &  Sponsor  &    Mean    & SD & Votes\\ \hline
\multicolumn{5}{l}{\underline{Woldenberg I}}   \\ [1.5ex]
C\'ardenas        & PRD &--1.79  &   0.44 & 230\\
Cant\'u           & PT  &  0.42  &   0.20 & 231\\
Zebad\'a          & PRD &  0.73  &   0.21 & 228\\
Lujambio          & PAN &  0.90  &   0.25 & 233\\
Molinar           & PAN &  1.09  &   0.26 & 238\\
Merino            & PRI &  1.95  &   0.45 & 244\\
Woldenberg        & PRI &  2.15  &   0.53 & 242\\
Peschard          & PRI &  2.28  &   0.60 & 242\\
Barrag\'an        & PRD &  3.25  &   1.03 & 204\\ [1ex]
\multicolumn{5}{l}{\underline{Woldenberg II}}  \\ [1.5ex]
C\'ardenas        & PRD &--1.67  &   0.23 & 290\\
Barrag\'an        & PRD &  0.40  &   0.12 & 246\\
Cant\'u           & PT  &  1.70  &   0.20 & 308\\
Luken             & PAN &  1.98  &   0.24 & 294\\
Rivera            & PRI &  3.20  &   0.38 & 318\\
Lujambio          & PAN &  3.50  &   0.45 & 323\\
Merino            & PRI &  3.60  &   0.44 & 330\\
Woldenberg        & PRI &  3.70  &   0.47 & 330\\
Peschard          & PRI &  3.75  &   0.44 & 323\\ [1ex]
\multicolumn{5}{l}{\underline{Ugalde}}         \\ [1.5ex]
Gonz\'alez Luna   & PAN &--2.61  &   0.47 & 145\\
S\'anchez         & PAN &--2.14  &   0.39 & 143\\
Albo              & PAN &--0.97  &   0.22 & 146\\
Latap\'i          & PRI &--0.47  &   0.17 & 146\\
Ugalde            & PRI &  0.23  &   0.16 & 141\\
L\'opez Flores    & PRI &  0.71  &   0.19 & 137\\
Andrade           & PRI &  1.09  &   0.25 & 146\\
Morales           & PAN &  1.24  &   0.27 & 143\\
G\'omez Alc\'antar& PVEM&  1.85  &   0.39 & 145\\ \hline
\end{tabular}
\end{center}
\end{table}

The partial turnover in council membership after 2000 resulted in some repositioning of member locations.  New entrants Luken and Rivera occupied Zebad\'ua's vacant slot between Councilors Cant\'u and Lujambio, while Molinar's departure left Councilors Lujambio and Merino as ideological neighbors.  The PRD's contingent in this council behaved more cohesively than before, with Barrag\'an leapfrogging toward the left.\footnote{In a generous reading, this councilor's 180-degree shift from the extreme right of the previous council reduced agency costs to his party sponsor.  Barrag\'an's behavior is so erratic, however, that it is nigh impossible to explain it within an ideological or partisan logic.  A two-dimensional rendering of ideal points helps make sense of this case, but we prefer to show results of a one-dimensional fit because it is simpler, and because degrees of overlap as predicted by the party influence hypothesis do not vary substantially in a two-dimensional model.} %\citep[cf.][]{Rosas2005a}.} 
Council members sponsored by the PRI continued to occupy the closely adjacent positions appropriate to bloc voting, but cohesion in the PAN contingent eroded.

%\footnote{Nominated by the PAN in 1996 as a substitute, Councilor Luken went on to take a position as Comptroller in the Federal District Government under PRD leadership before joining IFE in 2000.  The case is less one of inefficient screening than of unforeseeable co-sponsorship.  In that sense, his intermediate position between left-leaning colleagues and Councilor Lujambio is a plausible indicator of mixed partisan predispositions.} We interpret this as a reflection of changes in the issue-space that accompanied the replacement of two councilors. In the first place, the PRI contingent was enlarged by the turnover, which modified coalitional dynamics in the PRI's favor, inducing Lujambio to seek a tight-locked alliance on the right.  This change in voting power is reinforced by the unexpected salience of the dominant issues resolved under Woldenberg II, which involved charges of illegal campaign finance operations in 2000 against both the PAN and the PRI.\footnote{Former Councilor C\'ardenas consistently refused to join the majority in the resolutions on both controversies, and defends his minority position in a way that clarifies the distribution of preferences in the council, at least from 1996 to 2003 \citep{Cardenas2004}.}%  In his view, the fundamental division among council members concerned questions of citizen control versus vested partisan interests in electoral regulation, corresponding in our analysis to liberals versus conservatives. Of course, C�denas's persistent position on the liberal end of the spectrum for seven years dovetailed the ideological position of the PRD, his sponsoring party.}

Our party influence hypothesis continues to fare well after 2003.  Again, members of the PRI and PAN contingents are deployed in respectively adjacent positions with only one exception.  The new outlier is Councilor Morales near the conservative end of the spectrum, quite distant from his fellow PAN nominees. The ideological positions of Councilors Barrag\'an in the Woldenberg I and Morales in the Ugalde Councils are not consistent with the party sponsor hypothesis. However, eighteen councilors have ideological positions consistent with those of their sponsors, which suggests that parties are mostly able to reduce agency costs, either by screening \emph{ex ante} or signaling \emph{ex post}.

\begin{figure}
\begin{center}
  \caption{Ideology in IFE's Council-General (standardized range)}\label{F:ideolbars}
  \includegraphics[width=120mm]{"figure_2"}
\end{center}
\end{figure}

Also noteworthy is the finding that the posterior distributions of ideal points (which we also call ``ideal point ranges'') overlap in many instances.  This feature is easier to appreciate in Figure~\ref{F:ideolbars}, which shows the first-to-ninth-decile width of the posterior location parameter densities for each councilor in the three periods. These figures standardize the range of each council's ideological spectrum reported in Table~\ref{T:idealpoints} in order to facilitate the visual inspection of ideal points and ranges.\footnote{In the standardized spectrum, the left end of the left-most councilor's 80\% highest posterior density takes a value of 0, the right end of the right-most councilor's a value of 1.}  We hasten to add that neither the ideal points nor the measures of spread are strictly comparable across councils.  One can appreciate in the figures, for example, that PRI-sponsored council members in each half of the Woldenberg Council are virtually indistinguishable from each other, with overlapped ideal point ranges a sure sign of coalescent voting patterns in contested roll calls. Stacking of ranges in the Ugalde Council dropped significantly for all members, including those sponsored by the PRI.  The ideal point ranges of PAN nominees were stacked in the first half of the Woldenberg Council, but not afterwards. The ideal point ranges of PRD nominees  were never stacked. In the event, only four of eight multi-member contingents appear to exhibit the clustering of ideal point ranges that would signify fully consistent partisan bloc voting.  It is also interesting to note that member stacking dropped after the 2003 Council renewal. If bloc voting is a trait of partisanship, then the latest Council appears \emph{less} partisan than any of Woldenberg's Councils, which contradicts the view widely held during the campaign and post-election disputes that Ugalde's Council was inordinately partisan.

An even stronger statement of the party influence hypothesis would look to the formation of partisan cleavages based on bloc clustering.  We can address this expectation more systematically by performing ranks tests of the point estimates of councilors' ideologies in each council, using party sponsorship as the predictive categorical variable.  To the extent that significant inter-party differences are found in the positions of councilors, we could conclude that parties have succeeded in selecting representative agents whose like-mindedness undergirds partisan cleavages on the Council-General.

We report Kruskal-Wallis tests results in Table~\ref{T:kw}. Having only nine councilors forces us to consider only three of the four party contingents present in some Councils.\footnote{Statistical tables consulted do not report critical values for four-group small-sample tests \citep[][pp. 555-6]{Daniel1990}.} For both Woldenberg Councils, the test was performed twice: once dropping PT-sponsored Councilor Cant\'u from the sample; once categorizing him as PRD-sponsored. The null hypothesis that councilors sponsored by the PAN, the PRI, and the PRD come from identical populations can be rejected at levels below 0.01 for all councils (and regardless of how Mr. Cant\'u is handled). There is statistical evidence to support the claim that the populations from which parties handpick their appointees to the Council General have significantly different medians in the ideological spectrum. Even in the presence of ideological outliers in Woldenberg I (Barrag\'an) and Ugalde (Morales) it is possible to support the stronger version of the party influence hypothesis. Partisan segmentation of IFE would appear to be a fact of life.

\begin{table}
\caption{Ranks test of councilors' ideal points position by party
sponsorship}\label{T:kw}
\begin{center}
\begin{tabular}{lclcl}
\hline\\[-1.5ex]
Council       && Excluding Cant\'u  && Cant\'u as PRD     \\  \hline
Woldenberg I  && $T_{3,3,2}=16.0$   && $T_{4,3,2}=16.6$   \\
              && $Pr(>T)=0.011$     && $Pr(>T)=0.008$     \\
Woldenberg II && $T_{4,2,2}=31.1$   && $T_{4,3,2}=32.6$   \\
              && $Pr(>T)=0.011$     && $Pr(>T)=0.008$     \\
Ugalde        && $T_{4,4,1}=17.8$   &&                    \\
              && $Pr(>T)=0.010$     &&                    \\  \hline
\multicolumn{5}{l}{\footnotesize{{\sc Note}: All Kruskal-Wallis test statistics were between 3 and 5 times larger}}\\
\multicolumn{5}{l}{\footnotesize{than the greatest critical value reported by \citet[][Table A.12]{Daniel1990}.}}\\
\multicolumn{5}{l}{\footnotesize{True p-values are therefore substantially smaller than those we report.}}\\
\end{tabular}
\end{center}
\end{table}

Another way of approaching our results is by asking how probable it is to find same-party sponsors in adjacent ideological positions under the assumption of sponsor-independence---that is, as if such probabilities were not conditional upon party labels, like the null hypothesis claims. For Woldenberg I, this exercise is akin to drawing four yellow (PRD including Cant\'u), three white (PRI), and two blue (PAN) balls from an urn without replacement. Starting from the left of the spectrum, the (sponsor-independent) probability of getting three adjacent PRD balls is $\frac{4}{9} \times \frac{3}{8} \times \frac{2}{7} \approx 0.048$; that for three adjacent PRD balls and then two PAN balls is $0.048 \times \frac{2}{6} \times \frac{1}{5} \approx 0.003$.  The adjacency of ideological positions of councilors sponsored by the same party that we observe in all three councils for all three parties is a highly unlikely event---unless we believe that the events are \emph{not} sponsor-independent, as the party influence hypothesis suggests.

The mapping of subjacent ideological preferences in accordance with partisan sponsorship does not exhaust the voting data from IFE.
%\footnote{For example, we could use our ideal point estimates to gauge the probability that any one party might be decisive in IFE's Council-General by virtue of its control over the median councilor. By sampling a large number of times from the posterior distribution of ideal points and then counting the frequency with which each Councilor occupies the median, we are able to approximate this probability \citep{Clinton2004}. When we do so, we find that the PAN controlled the first half of the Woldenberg Council (we estimate the probability that the median Councilor was either Molinar or Lujambio as 0.834).  Council change in 2000 granted the PRI an opportunity to reclaim the median position.  In the second half of Woldenberg's Council, the probability that Councilors Merino or Rivera were median voters was 0.627, while there is only a 1-in-5 chance that Lujambio was the actual median voter. Finally, the PRI still seems in control of the rudder in Ugalde's Council, as we estimate the probability that either Ugalde or Latap�occupy the median as 0.626.  However, there is a non-negligible probability (0.31) that PAN-sponsored councilors Albo or S�chez currently occupy the median of Ugalde's Council.}
A fuller analysis of voting behavior on the Council-General must delve into the coalitional dynamics observed over time.  To the extent that councilors who are ideologically close can agree on common policy goals, the natural prediction is that they should coalesce in ideologically connected coalitions \citep{Axelrod1970}.  In spatial theory, when the status quo lies to the right of the median in a unidimensional spectrum, the left bloc votes together to bring policy towards the median member's ideal point, with coalition size (\emph{winsize}) increasing monotonically with the distance between the status quo and the median.  Table~\ref{T:cwcs} presents the aggregate evidence for connected majorities at IFE.  Note that in constructing this table we reverse our empirical strategy.  We first used roll-calls to infer ideological positions; we now use inferred ideologies to decide which of the observed voting coalitions are ideologically connected.\footnote{This strategy is commonly employed in the US congressional literature whenever {\sc Nominate} scores are used to predict vote choice.  \citet{Burden2000} show that {\sc Nominate} scores correlate highly with other indicators of ideology, even though they are built from observed roll-calls.}  In doing so, we do not ask whether inferred councilor ideologies account for individual voting patterns (by construction, our results are the ``best'' one-dimensional fit); instead, we ask how well our best model fits group voting patterns according to the criterion of ideological connectedness.

Several points in Table~\ref{T:cwcs} are worth highlighting.  First, even in the presence of extremists on either end of the spectrum, as in Woldenberg I, connected centrist coalitions have been exceedingly rare since 1996, which conforms to theoretical expectations for a one-dimensional spatial model of voting.  Second, each council shows a different pattern of connected coalition formation.  Woldenberg I alternated between oversized ``leftist'' and ``rightist''
coalitions.\footnote{We hasten to clarify that we employ ``leftist'' (respectively, ``rightist'') and ``from the left'' (respectively, ``right'') as synonymous, and strictly as directional qualifiers, i.e., we do not mean to imply that coalitions from the left represent the PRD's interests. Indeed, there are no PRD-sponsored councilors in Ugalde's Council, but we still refer to coalitions formed from the left.} Woldenberg II constructed majorities preponderantly from the right (comprising PRI and PAN contingents), while Ugalde has generated connected coalitions only from the left. Third, the proportion of unconnected majorities expands over time until they dominate contested roll-calls in the latest council. Over ten years, unconnected coalitions are smaller than connected ones by half a vote on average.  When non-extremist members drop out of a coalition, \emph{winsize} is reduced but the broad ideological range of the coalition remains constant.  Overall, fully 37\% of contested votes were decided by unconnected coalitions since 1996.

\begin{table}
\caption{Connected Winning Coalitions at IFE (mean size and frequency)}\label{T:cwcs}
\begin{center}
\begin{tabular}{lccccc}
\hline\\ [-1.5ex]
Council        & Leftwing & Centrist & Rightwing & Unconnected & Contested votes\\
               &  Winsize & Winsize & Winsize & Winsize & Winsize\\
               &  (\emph{Pct.})   &   (\emph{Pct.})   &  (\emph{Pct.})  & (\emph{Pct.})  & (N) \\ \hline \\ [-1ex]
Woldenberg I   &  7.51 &  6.37 &  7.28 &  6.51 & 7.13\\ [1ex]
               &  (\emph{28})  &   (\emph{3})  &  (\emph{45})  &  (\emph{24}) & (246)\\ [1.5ex]
Woldenberg II  &  6.00 &  6.40 &  7.23 &  6.83 & 7.05\\ [1ex]
               &  ($\mathit{<1}$)  &   (\emph{2})  &  (\emph{56})  &  (\emph{42}) & (336) \\ [1.5ex]
Ugalde         &  6.50 &   ---  &   ---  &  6.58 & 6.56\\ [1ex]
               &  (\emph{30})  &   (\emph{0})  &  (\emph{0})  &  (\emph{70}) & (54)\\ [1.5ex]
\hline
\end{tabular}
\end{center}
\end{table}

The direct implication of these patterns for the observation of partisan behavior by councilors is that coalitions at IFE, whether connected or not, tend to be cross-partisan and are inevitably so as \emph{winsize} increases.  But regardless of coalition size and coalitional dynamics, the underlying \emph{preference distribution} on the council nonetheless informs contested votes in a consistent and predictable fashion. The distribution of ideal points and their ranges props one inescapable conclusion: councilors are ideologically diverse but, with two notable exceptions, consistently aligned with their party sponsors.

\section{Conclusion}\label{S:discussion}
Notwithstanding the difficulties entailed by agenda control and powerful incentives toward consensual decisions which crowd out more narrowly partisan voting, we have detected important evidence of partisanship on IFE's Councils-General from 1996 to 2006.  The analysis of the posterior distribution of ideal points of these councilors provides evidence that nearly all Electoral Councilors have ideological preferences similar to those of other colleagues nominated by the same party sponsor. Moreover, there is evidence that council members grouped by party sponsor share ideal point ranges that cluster into discernible partisan blocs, distinct from one another. To that same extent, councilors are closer to their sponsors' hearts than might be expected in a putatively non-partisan electoral authority.

Our findings jibe with the view that legislative parties select IFE's Councils-General so as to retain control over agent behavior.  We believe our evidence is more than suggestive that though the bulk of IFE decisions have been above the political fray and free of partisan bickering, this is not because its members are embodiments of technocratic efficiency and impartiality.  Instead, the voting record is consistent with the view that councilors behave as party watchdogs, able to check each other's moves and assure compromises that protect their sponsors' interests in the electoral arena. Thus, the paradox of Mexico's success story, at least until 2006, is that IFE has been and remains a \emph{non-autonomous agent}. Parties have not given away full control of the levers of electoral regulation and yet benefit from the reputation that elections are clean.

However, for this power-sharing model to work, all major parties should be represented. The absence of the PRD in the enacting coalition that named the most recent Council-General was cause for concern that materialized in the aftermath of the 2006 presidential race. Because parties anticipate that their interests will be guarded by their sponsored council members and can be reasonably sure that agency losses will be minor, they are willing to obey the occasional ruling that hurts their short-term interests. Parties may more adamantly oppose electoral regulation if they suspect that their preferences will not receive a fair hearing. In short, analysis has shown that Mexico's electoral regulator, which has enjoyed high levels of trust by citizens, has manifested partisan behavior within its ranks since its inception.   Our results invite cross-national research to support the broader theoretical claim that election arbiters that embrace partisan strife, rather than those that purport to expunge party politics altogether from electoral regulation, are better able to guarantee free and fair elections in new democracies.

In general terms, the behavioral patterns that our investigation has uncovered have been observed in other, more consolidated, democracies.  Johnston \citeyearpar{Johnston2002} detects robust signs of pro-Tory and pro-Labour bias in U.K. Parliamentary constituencies since 1950, despite redistricting plans prepared by ``independent'' Boundary Commissions chaired by a senior Judge. Casual observation of other EMBs suggests that the partisan logic that we have documented in Mexico's IFE is not unknown elsewhere. In describing the ``politics of co-participation'' that characterize the Uruguayan party-based \emph{Corte Electoral}, \citeauthor{LopezPintor2000} recognizes this point: ``Whatever negotiations take place in the political arena influence the Corte's decisions; conversely, decisions adopted by the Corte can easily be assumed by the parties as their own. This applies to both informal politics and to law-making. Consequently, once a decision is taken, all the parties usually implement it'' \citep[p.  23]{LopezPintor2000}. \citeauthor{Mozaffar2002a} echo this point in their description of electoral bodies built as quasi-consociational schemes of power-sharing \citep[p. 16]{Mozaffar2002a}.

Miller \citeyearpar{Miller2005} uncovers a whole class of situations where the fundamental problem of striking credible commitments is solved through delegation to agents that remain, to a large extent, autonomous from principals. Our conjecture is that checks-and-balances arrangements should suffice in such situations.  Our discussion and findings relate to classic studies of the politics of gerrymandering, where a debate has long been waged between proponents of non-partisan boards for redistricting and those favoring co-partisan arrangements for reducing the incidence of partisan bias (see \citet{ButlerCain1992} for the U.S. and
Australian cases, \citet{RossiterEtal1998, Rossiter1997} for the U.K. and Northern Ireland). They should also stimulate debate farther away from the area of electoral regulation: the watchdog scheme has been the model of choice to handle ethnic strife in deeply divided societies, as evidenced by the United Nations' Minority Police Recruitment program set in place in Bosnia and Herzegovina in the second half of the 1990s to organize multi-ethnic police forces \citep{Collantes2005}.


\pagebreak
\bibliographystyle{apsr}

\begin{thebibliography}{77}

\harvarditem{Alcocer}{1995}{Alcocer1995}
Alcocer, Jorge. 1995.
\newblock 1994: {D}i{\'a}logo y reforma, un testimonio.  In {\em Elecciones,
  di{\'a}logo y reforma: {M}{\'e}xico 1994}, ed. Jorge Alcocer~V.
\newblock Vol.~I Mexico City:  Nuevo Horizonte.

\harvarditem{Axelrod}{1970}{Axelrod1970}
Axelrod, Robert. 1970.
\newblock {\em Conflict of Interest: A Theory of Divergent Goals with
  Applications to Politics}.
\newblock Chicago, IL:  Markham.

\harvarditem[Burden, Caldeira \harvardand\ Groseclose]{Burden, Caldeira
  \harvardand\ Groseclose}{2000}{Burden2000}
Burden, Barry~C., Gregory~A. Caldeira \harvardand\ Tim Groseclose. 2000.
\newblock ``Measuring the Ideologies of {U.S.} {S}enators: The Song Remains the
  Same.'' {\em Legislative Studies Quarterly} 25(2):237--258.

\harvarditem{Butler \harvardand\ Cain}{1992}{ButlerCain1992}
Butler, David \harvardand\ Bruce Cain. 1992.
\newblock {\em Congressional Redistricting}.
\newblock New York, NY:  MacMillan.

\harvarditem[Cameron, Segal \harvardand\ Songer]{Cameron, Segal \harvardand\
  Songer}{2000}{Cameron2000}
Cameron, Charles~M., Jeffrey~A. Segal \harvardand\ Donald~R. Songer. 2000.
\newblock ``Strategic Auditing in a Political Hierarchy: An Informational Model
  of the Supreme Court's Certiorari Decisions.'' {\em American Political
  Science Review} 94(1):101--116.

\harvarditem{C{\'a}rdenas}{2004}{Cardenas2004}
C{\'a}rdenas, Jaime. 2004.
\newblock {\em Lecciones de los asuntos Pemex y Amigos de Fox}.
\newblock Number 186 {\em in} ``Doctrina Jur{\'\i}dica'' UNAM-IIJ.

\harvarditem{Clayton}{2002}{Clayton2002}
Clayton, Cornell~W. 2002.
\newblock ``The Supply and Demand Sides of Judicial Policy-Making (or Why Be So
  Positive about the Judicialization of Politics?).'' {\em Law and Contemporary
  Problems} 65(3):69--86.

\harvarditem[Clinton, Jackman \harvardand\ Rivers]{Clinton, Jackman
  \harvardand\ Rivers}{2004}{Clinton2004}
Clinton, Joshua, Simon Jackman \harvardand\ Douglas Rivers. 2004.
\newblock ``The {S}tatistical {A}nalysis of {R}oll {C}all {D}ata.'' {\em
  American Political Science Review} 98(2):355--370.

\harvarditem{Collantes~Celador}{2005}{Collantes2005}
Collantes~Celador, Gemma. 2005.
\newblock ``Police reform: Peacebuilding through 'democratic policing'?'' {\em
  International Peacekeeping} 12(3):364--76.

\harvarditem{Cox \harvardand\ McCubbins}{1993}{Cox1993}
Cox, Gary~W. \harvardand\ Mathew~D. McCubbins. 1993.
\newblock {\em Legislative Leviathan}.
\newblock Berkeley:  University of California Press.

\harvarditem{Daniel}{1990}{Daniel1990}
Daniel, Wayne~W. 1990.
\newblock {\em Applied Nonparametric Statistics}.
\newblock Second ed. Boston:  PWS-Kent.

\harvarditem{Eisenstadt}{2004}{Eisenstadt2004}
Eisenstadt, Todd~A. 2004.
\newblock {\em Courting Democracy in {M}exico. Party Strategies and Electoral
  Institutions}.
\newblock New York, NY:  Cambridge University Press.

\harvarditem{Estrada \harvardand\ Poir\'e}{2007}{Estrada2007}
Estrada, Luis \harvardand\ Alejandro Poir\'e. 2007.
\newblock ``Learning to {L}ose, {T}aught to {P}rotest: {M}exico's 2006
  {E}lection.'' {\em Journal of Democracy} 18(1).

\harvarditem{Gely \harvardand\ Spiller}{1990}{GelySpiller1990}
Gely, Rafael \harvardand\ Pablo~T. Spiller. 1990.
\newblock ``A Rational Choice Theory of Supreme Court Statutory Decisions with
  Applications to the \emph{State Farm} and \emph{Grove City} Cases.'' {\em
  Journal of Law, Economics, and Organization} 6(2):263--300.

\harvarditem[Hamilton, Jay \harvardand\ Madison]{Hamilton, Jay \harvardand\
  Madison}{1788}{Madison1788}
Hamilton, Alexander, John Jay \harvardand\ James Madison. 1788.
\newblock {\em The Federalist}.
\newblock J. \& A. McLean.

\harvarditem[Hartlyn, McCoy \harvardand\ Mustillo]{Hartlyn, McCoy \harvardand\
  Mustillo}{forthcoming}{Hartlyn2006}
Hartlyn, Jonathan, Jennifer McCoy \harvardand\ Thomas~M. Mustillo. forthcoming.
\newblock ``Electoral Governance Matters: {E}xplaining the Quality of Elections
  in Contemporary {L}atin {A}merica.'' {\em Comparative Political Studies} .

\harvarditem{Huber \harvardand\ Shipan}{2002}{Huber2002}
Huber, John~D. \harvardand\ Charles~R. Shipan. 2002.
\newblock {\em Deliberate Discretion? The Institutional Foundations of
  Bureaucratic Autonomy}.
\newblock Cambridge Studies in Comparative Politics Cambridge, UK:  Cambridge
  University Press.

\harvarditem{Jackman}{2001}{Jackman2001}
Jackman, Simon. 2001.
\newblock ``Multidimensional Analysis of Roll Call Data via {B}ayesian
  Simulation: Identification, Estimation, Inference, and Model Checking.'' {\em
  Political Analysis} 9(3):227--241.

\harvarditem{Johnston}{2002}{Johnston2002}
Johnston, Ron. 2002.
\newblock ``Manipulating Maps and Winning Elections: Measuring the Impact of
  Malapportionment and Gerrymandering.'' {\em Political Geography} 21(1):1--31.

\harvarditem{Jones}{1995}{Jones1995}
Jones, Mark~P. 1995.
\newblock {\em Electoral {R}ules and the {S}urvival of {P}residential
  {R}egimes}.
\newblock Notre Dame, IN:  University of Notre Dame Press.

\harvarditem{Kiewiet \harvardand\ McCubbins}{1991}{Kiewiet1991}
Kiewiet, Roderick \harvardand\ Mathew~D. McCubbins. 1991.
\newblock {\em The Logic of Delegation}.
\newblock University of Chicago Press.

\harvarditem{Lax}{2003}{Lax2003}
Lax, Jeffrey~R. 2003.
\newblock ``Certiorari and Compliance in the Judicial Hierarchy.'' {\em Journal
  of Theoretical Politics} 15(1):61--86.

\harvarditem{Linz \harvardand\ Stepan}{1978}{Linz1978}
Linz, Juan \harvardand\ Alfred Stepan. 1978.
\newblock {\em The Breakdown of Democratic Regimes}.
\newblock Baltimore, MD:  The Johns Hopkins University Press.

\harvarditem{Little \harvardand\ Rubin}{1987}{Little1987}
Little, Roderick~J. \harvardand\ Donald~B. Rubin. 1987.
\newblock {\em Statistical Analysis with Missing Data}.
\newblock John Wiley \& Sons.

\harvarditem{Londregan}{2000}{Londregan2000}
Londregan, John. 2000.
\newblock {\em Legislative Institutions and Ideology in {C}hile's Democratic
  Transition}.
\newblock New York, NY:  Cambridge University Press.

\harvarditem{L\'opez-Pintor}{2000}{LopezPintor2000}
L\'opez-Pintor, Rafael. 2000.
\newblock {\em Electoral Management Bodies as Institutions of Governance}.
\newblock United Nations Development Programme.

\harvarditem{Lujambio}{2001}{Lujambio2001}
Lujambio, Alonso. 2001.
\newblock Adi\'os a la excepcionalidad: {R}\'egimen presidencial y gobierno
  dividido en {M}\'exico.  In {\em Tipos de presidencialismo y coaliciones
  pol\'iticas en {A}m\'erica {L}atina}, ed. Jorge Lanzaro.
\newblock CLACSO.

\harvarditem{Magaloni}{2006}{Magaloni2006}
Magaloni, Beatriz. 2006.
\newblock {\em Voting for {A}utocracy: {H}egemonic {P}arty {S}urvival and its
  {D}emise in {M}exico}.
\newblock Cambridge University Press.

\harvarditem{Malo \harvardand\ Pastor}{1996}{Malo1996}
Malo, Ver\'onica \harvardand\ Julio Pastor. 1996.
\newblock {\em Autonom\'ia e imparcialidad en el {C}onsejo {G}eneral del {IFE},
  1994-1995}.
\newblock Unpublished thesis Departamento de Ciencia Pol\'itica:  Instituto
  Tecnol\'ogico Aut\'onomo de M\'exico.

\harvarditem{Martin \harvardand\ Quinn}{2002}{Martin2002}
Martin, Andrew~D. \harvardand\ Kevin~M. Quinn. 2002.
\newblock ``Dynamic {I}deal {P}oint {E}stimation via {M}arkov {C}hain {M}onte
  {C}arlo for the {U}.{S}. {S}upreme {C}ourt, 1953-1999.'' {\em Political
  Analysis} 10(2):134--153.

\harvarditem{McCubbins \harvardand\ Page}{1987}{McCubbins1987}
McCubbins, Mathew~D. \harvardand\ Talbot Page. 1987.
\newblock A Theory of Congressional Delegation.  In {\em Congress: Structure
  and Policy}, ed. Mathew~D. McCubbins \harvardand\ Terry Sullivan.
\newblock Cambridge University Press pp.~409--25.

\harvarditem[McCubbins, Noll \harvardand\ Weingast]{McCubbins, Noll
  \harvardand\ Weingast}{1999}{Mcnollgast1999}
McCubbins, Matthew, Roger Noll \harvardand\ Barry Weingast. 1999.
\newblock ``The Political Origins of the Administrative Procedure Act.'' {\em
  Journal of Law, Economics, and Organization} 15(1):180--217.

\harvarditem{Merino}{1999}{Merino1999}
Merino, Mauricio. 1999.
\newblock ``Divisiones p\'ublicas, consensos internos.'' {\em Enfoque.
  Suplemento de Reforma} .

\harvarditem{Miller}{2005}{Miller2005}
Miller, Gary~J. 2005.
\newblock ``The Political Evolution of Principal-Agent Models.'' {\em Annual
  Review of Political Science} 8:203--225.

\harvarditem{Molina \harvardand\ Hern\'andez}{1999}{Molina1999}
Molina, Jos\'{e} \harvardand\ Janeth Hern\'andez. 1999.
\newblock ``La credibilidad de las elecciones latinoamericanas y sus factores.
  {E}l efecto de los organismos electorales, el sistema de partidos y las
  actitudes politicas.'' {\em Cuadernos del CENDES} 41:1--26.

\harvarditem{Molinar~Horcasitas}{1991}{Molinar1991}
Molinar~Horcasitas, Juan. 1991.
\newblock {\em El tiempo de la legitimidad}.
\newblock Mexico City:  Cal y Arena.

\harvarditem{Moreno}{2003}{Moreno2003a}
Moreno, Alejandro. 2003.
\newblock {\em El votante mexicano. Democracia, actitudes pol\'iticas y
  conducta electoral}.
\newblock M\'exico, DF:  Fondo de Cultura Econ\'omica.

\harvarditem{Mozaffar}{2002}{Mozaffar2002}
Mozaffar, Shaheen. 2002.
\newblock ``Patterns of Electoral Governance in Africa's Emerging
  Democracies.'' {\em International Political Science Review} 23(1):85--101.

\harvarditem{Mozaffar \harvardand\ Schedler}{2002}{Mozaffar2002a}
Mozaffar, Shaheen \harvardand\ Andreas Schedler. 2002.
\newblock ``The Comparative Study of Electoral Governance---Introduction.''
  {\em International Political Science Review} 23(1):5--27.

\harvarditem{Przeworski}{1991}{Przeworski1991}
Przeworski, Adam. 1991.
\newblock {\em Democracy and the Market: Political and economic reforms in
  Eastern Europe and Latin America}.
\newblock Cambridge, MA:  Cambridge University Press.

\harvarditem{Putnam}{2000}{Putnam2000}
Putnam, Robert. 2000.
\newblock {\em Bowling {A}lone: {T}he {C}ollapse and {R}evival of {A}merican
  {C}ommunity}.
\newblock New York:  Simon \& Schuster.

\harvarditem[Rossiter, Johnston \harvardand\ Pattie]{Rossiter, Johnston
  \harvardand\ Pattie}{1997}{Rossiter1997}
Rossiter, D.J., R.J. Johnston \harvardand\ C.J. Pattie. 1997.
\newblock ``Estimating the Partisan Impact of Redistricting in {G}reat
  {B}ritain.'' {\em British Journal of Political Science} 27(2):319--331.

\harvarditem[Rossiter, Johnston \harvardand\ Pattie]{Rossiter, Johnston
  \harvardand\ Pattie}{1998}{RossiterEtal1998}
Rossiter, D.J., R.J. Johnston \harvardand\ C.J. Pattie. 1998.
\newblock ``The Partisan Impacts of Non-Partisan Redistricting: Northern
  Ireland, 1993-95.'' {\em Transactions of the Institute of British
  Geographers, New Series} 23(4):455--80.

\harvarditem{Schedler}{2000}{Schedler2000a}
Schedler, Andreas. 2000.
\newblock ``Incertidumbre institucional e inferencias de imparcialidad: {E}l
  caso del {I}nstituto {F}ederal {E}lectoral.'' {\em Pol\'{i}tica y Gobierno}
  7(2):383--421.

\harvarditem[Schedler, Diamond \harvardand\ Plattner]{Schedler, Diamond
  \harvardand\ Plattner}{1999}{Schedler1999}
Schedler, Andreas, Larry Diamond \harvardand\ Marc~F. Plattner. 1999.
\newblock {\em The Self-Restraining State: Power and Accountability in New
  Democracies}.
\newblock Lynne Rienner Publishers.

\harvarditem[Songer, Segal \harvardand\ Cameron]{Songer, Segal \harvardand\
  Cameron}{1994}{Songer1994}
Songer, Donald~R., Jeffrey~A. Segal \harvardand\ Charles~M. Cameron. 1994.
\newblock ``The Hierarchy of Justice: Testing a Principal-Agent Model of
  Supreme Court-Circuit Court Interactions.'' {\em American Journal of
  Political Science} 38(3):673--696.

\end{thebibliography}

\end{document}
\pagebreak
\appendix
\section*{Web appendix [not meant for publication]}\label{S:model}

Political scientists rely on the Euclidean spatial model to build up their analyses of committee voting from solid first principles (Ordeshook 1976, Hinich \& Munger 1994).  Put succinctly, spatial models assume that, when facing a binary YEA or NAY vote choice, rational committee members will vote for the alternative that will enact the policy closest to their own ideal position.  \citet{Clinton2004} formalize this utility calculation as follows (see also \citet{Jackman2001, Martin2002}): Let $U_{i}(\zeta_{j})= - \norm{x_{i}-\zeta_{j}}^{2}+\eta_{i,j}$ represent the utility to committee member $i \in I_{n}$ of voting in favor of proposal $j \in J_{m}$ and $U_{i}(\psi_{j})= -\norm{x_{i}-\psi_{j}}^{2} + \nu_{i,j}$ the utility of voting against it.

In this formalization of the one dimensional model, $x_{i}$, $\zeta_{j}$, and $\psi_{j}$ correspond, respectively, to the ideal position of the committee member, the position that will result from a YEA vote, and the position that will result from a NAY vote.  Disturbances $\eta_{i,j}$ and $\nu_{i,j}$ are assumed to be distributed bivariate normal with zero means and known variance; they are also assumed uncorrelated.

To turn the formal utility representation into a statistical model susceptible of estimation, note that a positive vote by member $i$ on proposal $j$ ($y_{i,j}=1$) reveals that $U_{i}(\zeta_{j})$ $ \geq U_{i}(\psi_{j})$.  It follows that a committee member will vote YEA on any given proposal if $U_{i}(\zeta_{j}) - U_{i}(\psi_{j}) > 0$: 

\begin{align}\label{E:equation1}
y_{i,j}
   &= U_{i}(\zeta_{j} ) - U_{i}(\psi_{j}) \\
   &=  -\norm{x_{i}-\zeta_{j}}^{2}+\eta_{i,j} +\norm{x_{i}-\psi_{j}}^{2}+\nu_{i,j} \nonumber \\
   &=  2(\eta_{j}-\psi_{j})x_{i} + \psi_{j}^{2}-\zeta_{j}^{2}+ \eta_{i,j} +\nu_{i,j} \nonumber \\
   &= \alpha_{j} + \beta_{j} x_{i} + \varepsilon_{i,j}, \nonumber
\end{align}

\noindent where $\alpha_{j}=\psi_{j}^{2}-\zeta_{j}^{2}$, $\beta_{j}= 2(\eta_{j}-\psi_{j})$, and $\varepsilon_{i,j}=\eta_{i,j} +\nu_{i,j}$.  The last line in Equation~(\ref{E:equation1}) can be rearranged to represent each vote $y_{i,j}$ as an independent draw from a normal probability distribution; thus $p(y_{i,j}=1) = \int_{0}^{\infty} \Phi(\alpha_{j}+\beta_{j} x_{i})$, where $\Phi(\cdot)$ is the normal cumulative distribution function.   If, for notational convenience, the parameters $\alpha_{j}$, $\beta_{j}$, and $x_{i}$ are stacked in vectors $\boldsymbol{\alpha}$, $\boldsymbol{\beta}$, and $\mathbf{x}$ (of lengths $m$, $m$, and $n$ respectively), the likelihood function can be constructed from the observed $\mathbf{Y}$:

\begin{equation}\label{E:equation2}
\mathcal{L}(\boldsymbol{\alpha},\boldsymbol{\beta},\mathbf{x}|\mathbf{y})
= \prod_{j=1}^{m} \prod_{i=1}^{n}  \Phi(\alpha_{j}+\beta_{j}
x_{i})^{y_{i,j}} (1-\Phi(\alpha_{j}+\beta_{j} x_{i}))^{1-y_{i,j}}
\end{equation}

The likelihood function in Equation~(\ref{E:equation2}) can be estimated statistically.  Note however that we require estimates of $\boldsymbol{\alpha}$ and $\boldsymbol{\beta}$ (the item, case, or bill parameters), and $\mathbf{x}$ (the ideal points of councilors), and that we only have information collected in the matrix $\mathbf{Y}$ of observed votes (0's and 1's) for all committee members on all proposals discussed by IFE's Council-General.  As it stands, thus, the model is not identified, because $\boldsymbol{\alpha}$, $\boldsymbol{\beta}$, and $\mathbf{x}$ admit an infinite number of solutions.\footnote{There are two sources of under-identification in item response models: scale invariance and rotational invariance (\citet{Jackman2001} offers an excellent discussion of identification problems in two-dimensional models).}
Thus, in order to allow identification of the model parameters, it is necessary to add restrictions on their possible values.  One can alternatively fix $\mathbf{x}_{i}$ for ``known'' holders of extreme views in the committee, or fix the $\boldsymbol{\beta}$ discrimination parameters for some bills or decisions.  As explained in the text, we prefer the latter approach.  We solve the problem of rotational invariance by fixing $\boldsymbol{\beta}$ for two votes.  By stipulating prior distributions for councilors' positions with identical variance, we solve the problem of scale invariance.  In the Bayesian approach, these prior distributions are combined with the likelihood function in (\ref{E:equation2}) to obtain the joint posterior distribution of the parameters of interest.

The identification restrictions on the item parameters are detailed in Table~\ref{T:priors}.  For example, for the Woldenberg II Council (2000-2003) we imposed restrictions on the discrimination parameters of votes 85 and 207 (see Table~\ref{T:priors}) to construct a common one-dimensional space within which we could locate the ideological positions of Electoral councilors.  In all councils, the discrimination parameters of the items that we chose for identification are restricted as follows:

\begin{align*}\label{E:specialpriors}
\beta_{L} &\sim \mathcal{N}(-4, 4) \\ \nonumber
\beta_{R} &\sim \mathcal{N}(4, 4)
\end{align*}

\noindent where subindex $L$ indicates the vote we chose to anchor the one-dimensional space on the left and $R$ is the right anchor.  For all other discrimination parameters $\boldsymbol{\beta}$ and all difficulty parameters $\boldsymbol{\alpha}$ we stipulate prior standard normal distributions.

As an \emph{ex post} check on the appropriateness of the votes we selected to anchor the ideological spaces, Table~\ref{T:identification} presents posterior means and standard deviations for the discrimination parameters of these votes. We find that the posterior means of these parameters are closer to ``0'' than we had stipulated in our prior distributions, but the distributions of left (right) anchors clearly remain on the negative (positive) orthant, thus convincing us that we chose votes that appropriately discriminate between left and right.

\begin{table}
\caption{Posterior distribution of identification parameters}\label{T:identification}
\begin{center}
\begin{tabular}{lrr}
\hline
Parameter  &  Mean  &  SD \\ \hline
\multicolumn{3}{l}{\underline{Woldenberg I}}\\
$\beta_{28}$     &--1.67  &   0.79\\

$\beta_{228}$    &  1.66  &   0.78\\ [1.2ex]
\multicolumn{3}{l}{\underline{Woldenberg II}}  \\
$\beta_{85}$    &  0.69  &   0.35\\
$\beta_{207}$   &--0.94  &   0.40\\ [1.2ex]
\multicolumn{3}{l}{\underline{Ugalde}}  \\
$\beta_{33}$    &  2.73  &   1.42\\
$\beta_{43}$    &--4.60  &   1.67\\ \hline
\end{tabular}
\end{center}
\end{table}

We estimate the joint posterior distribution with the Gibbs sampling algorithm in WinBugs.  For each of our two datasets, we ran 200,000 iterations of the Gibbs sampler, discarding 100,000 as burn-in and thinning the resulting chain so as to keep 10,000 draws from the posterior distribution for inference purposes. We monitored convergence through Geweke's statistics.  Samples and convergence results are available for inspection from the authors.


kruskal-wallis
Wold 1 EXCLUDING CANTU
    pan pri prd all
    3   5   1
    4   6   2
        7   8
ni  2   3   3   8
si  7   18  11
Sti 24.5    108 40.33333333
St^2 172.8333333
Sr^2    204
T   16.03571429 p<  0.011

WOLD 1 CANTU AS A PRD
    pan pri prd-pt  all
    4   6   1
    5   7   2
        8   3
            9
ni  2   3   4   9
si  9   21  15
Sti 40.5    147 56.25
St^2    243.75
Sr^2    285
T   16.625  p<  0.008

Wold 2 EXCLUDING CANTU
    pan pri prd all
    3   4   1
    5   6   2
        7
        8
ni  2   3   3   8
si  8   25  3
Sti 32  208.3333333 3
St^2 243.3333333
Sr^2    204
T   31.14285714 p<  0.011

Wold 2 CANTU AS PRD
    pan pri prd all
    4   5   1
    6   7   2
        8   3
        9
ni  2   3   4   9
si  10  29  6
Sti 50  280.3333333 9
St^2    339.3333333
Sr^2    285
T   32.55555556 p<  0.008

UGALDE
    pan pri pvem all
    1   4   9
    2   5
    3   6
    8   7
ni  4   4   1   9
si  14  22  9
Sti 49  121 81
St^2    251
Sr^2 285
T   17.83333333 p<  0.010






anova
Wolden I
       BSS/TSS           TSS              BSS               WSS              BMS               WMS
 Min.   :0.004981   Min.   : 4.957   Min.   : 0.1148   Min.   : 2.656   Min.   :0.05741   Min.   :0.5313
 1st Qu.:0.118376   1st Qu.:13.966   1st Qu.: 2.1317   1st Qu.: 9.668   1st Qu.:1.06584   1st Qu.:1.9336
 Median :0.188187   Median :17.644   Median : 3.3005   Median :14.098   Median :1.65026   Median :2.8195
 Mean   :0.225782   Mean   :17.985   Mean   : 3.7289   Mean   :14.256   Mean   :1.86443   Mean   :2.8511
 3rd Qu.:0.311414   3rd Qu.:21.495   3rd Qu.: 4.9012   3rd Qu.:18.152   3rd Qu.:2.45059   3rd Qu.:3.6304
 Max.   :0.778114   Max.   :39.027   Max.   :15.2570   Max.   :38.146   Max.   :7.62849   Max.   :7.6292




Wolden II
       BSS/TSS        TSS             BSS             WSS              BMS              WMS
 Min.   :0.7103   Min.   :18.18   Min.   :15.35   Min.   : 1.707   Min.   : 7.674   Min.   :0.3415
 1st Qu.:0.8331   1st Qu.:25.66   1st Qu.:21.93   1st Qu.: 3.343   1st Qu.:10.966   1st Qu.:0.6687
 Median :0.8554   Median :27.93   Median :23.78   Median : 4.027   Median :11.888   Median :0.8055
 Mean   :0.8527   Mean   :28.08   Mean   :23.93   Mean   : 4.157   Mean   :11.963   Mean   :0.8313
 3rd Qu.:0.8758   3rd Qu.:30.25   3rd Qu.:25.74   3rd Qu.: 4.792   3rd Qu.:12.870   3rd Qu.:0.9584
 Max.   :0.9388   Max.   :45.05   Max.   :38.89   Max.   :10.849   Max.   :19.445   Max.   :2.1697




Ugalde
       BSS/TSS            TSS              BSS                 WSS              BMS                 WMS
 Min.   :9.251e-07   Min.   : 1.849   Min.   :7.801e-06   Min.   : 1.549   Min.   :3.901e-06   Min.   :0.3098
 1st Qu.:7.508e-02   1st Qu.: 5.251   1st Qu.:4.577e-01   1st Qu.: 4.363   1st Qu.:2.289e-01   1st Qu.:0.8725
 Median :1.435e-01   Median : 6.706   Median :9.319e-01   Median : 5.623   Median :4.660e-01   Median :1.1247
 Mean   :1.561e-01   Mean   : 7.079   Mean   :1.127e+00   Mean   : 5.953   Mean   :5.634e-01   Mean   :1.1905
 3rd Qu.:2.255e-01   3rd Qu.: 8.520   3rd Qu.:1.556e+00   3rd Qu.: 7.191   3rd Qu.:7.781e-01   3rd Qu.:1.4382
 Max.   :6.098e-01   Max.   :19.251   Max.   :7.086e+00   Max.   :18.415   Max.   :3.543e+00   Max.   :3.6830


