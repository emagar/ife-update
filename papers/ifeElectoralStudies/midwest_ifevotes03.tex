%%% IGNORE ALL THIS FRONT MATTER
\documentclass[12 pt]{amsart}
\usepackage{amssymb,latexsym,amsfonts,amsmath,ctable}
\usepackage{graphicx}
\renewcommand{\bibitem}{\vskip2pt\par\hangindent\parindent\hskip-\parindent}
\newcommand{\norm}[1]{\lVert#1\rVert}
\begin{document}
\title[Party sponsorship in IFE]{Party sponsorship and voting behavior in small committees: Mexico's \emph{Instituto Federal Electoral}}
\author[Rosas]{Guillermo Rosas}
%\address{Dept. of Political Science\\
%Washington University}
\author[Est\'evez]{Federico Est\'evez}
%\address{Dept. of Political Science\\ITAM}
\author[Magar]{Eric Magar}
%\address{Dept. of Political Science\\ITAM}
%\address{Department of Political Science\\Washington University in St. Louis\\St. Louis, MO 63130}
%\email{grosas@wustl.edu}
%\urladdr{http://poli.wustl.edu/grosas/}
\thanks{We are deeply grateful to Alonso Lujambio and Jeff Weldon for comments and suggestions, and to Mariana Medina, Sergio Holgu\'in, and Gustavo Robles for superb research assistance.  Thanks also to the Weidenbaum Center at Washington University in St. Louis for its generous support.}
%\date{\today}
%%% THE DOCUMENT STARTS HERE, WITH THE ABSTRACT
\begin{abstract}
We infer the ideological positions of Citizen Councilors in Mexico's \emph{Instituto Federal Electoral} (IFE) using Bayesian methods appropriate to the analysis of small committees.  We determine whether inferred ideological stances are consistent with a partisan sponsorship interpretation of IFE's institutional setup or, conversely, if voting behavior is largely supra-partisan. The analysis is based on votes cast by 11 members of IFE's second Council General, from 1996 to 2003.
\end{abstract}
\maketitle

\section{Introduction}\label{S:introduction}
During elections in 1997 and 2000, the Mexican citizenry ousted the party that held power uninterrupted for nearly seven decades.  They did so peacefully, in the ballot boxes.  In the aftermath of these foundational elections, much of the credit for the success of the Mexican transition to democracy has gone to the authority in charge of planning and executing all electoral matters, namely, the Federal Electoral Institute (IFE).  A new Council General ---IFE's board of directors--- had been appointed in 1996; this Council oversaw the midterm election of 1997, when the ruling Institutional Revolutionary Party (PRI) lost control of the lower chamber of Congress, and the presidential election of 2000, when the right-of-center National Action Party (PAN) defeated the PRI.

Mexico is not alone in having set up an \emph{electoral management body} (EMB) to manage democracy.\footnote{Info on that website} Unfortunately, the scientific literature on EMBs is rather sparse.  We do not know what factors determine their establishment or their degree of autonomy, nor we know much about the consequences that follow from alternative EMB arrangements.  To our knowledge, only Molina \& Hern\'andez (1999) have explored whether citizens' trust in the electoral process is in any way affected by the institutional character of an EMB.  According to these authors, EMBs vary first and foremost on the political principle that they seek to embody.  Some EMBs are organized according to a \emph{principle of impartiality}.  These organizations tend to be run by non-partisan technocrats selected through a meritocratic process.  Other EMBs are designed according to the \emph{principle of checks and balances}, granting political actors the ability to designate representatives in the hope that ``party watchdogs'' will rabidly keep tabs on each other's behavior.\footnote{Molina \& Hern\'andez (1999) marshal some evidence suggesting that EMBs organized according to the principle of impartiality tend to elicit more trust among citizens.}

Prima facie, IFE Citizen Councilors personify non-partisan technocratic efficiency.  They are thoroughly vetted and recruited from a select set of professionals without party affiliations and admitted to IFE's Council only after winning the endorsement of majorities in the Mexican Congress and Senate. Furthermore, IFE's operative budget is not subject to political whims.  At the same time, however, Congressional parties are the only agents that can nominate candidates to the Council.  Moreover, as we explain below, the post-Council careers of some Citizen Councilors suggest continuous links with political parties.  These features lead us to believe that Councillors might be more attuned to the goals of their party sponsors than one would surmiss from their lack of explicit party affiliations.

In this paper, we study the voting patterns of IFE's Citizen Councilors in the crucial period between October 1996 and December 2003. In Section \ref{S:description}, we identify several aspects of IFE's institutional setup that lead to conflicting hypotheses about Councilors' voting behavior.  We purport to decide empirically whether this behavior is consistent with the ``technocratic'' or ``party-sponsorship'' interpretations of EMBs.  We do so in Section \ref{S:estimation}, where we use Bayesian MCMC estimation methods to infer the ideal points of IFE Councilors in both one- and two-dimensional ideological spaces.  We conclude in Section \ref{S:discussion} with suggestions for further research.

\section{IFE's Institutional Setup:  Hypotheses}\label{S:description}
IFE was established in the early 1990's as a semi-autonomous bureaucratic agency in charge of overseeing federal elections. Though IFE's charter originally called for a preponderant presence of the Executive power in its board, successive reforms led to the creation of a vigorous agency independent from Mexico's once omnipotent Presidents. Concurrent with its increasing autonomy, IFE took over the years an ever more important role in organizing all electoral aspects of Mexico's protracted transition to democracy. IFE's Council General decides on all organizational matters relating to elections, including the creation and upkeep of electoral lists, installation of electoral booths, vote counting, monitoring of campaign spending by parties, and overall regulation of political campaigns.

In its earliest incarnation (1994-1996), IFE's Council General included ten Councilors. Four of them were Legislative Councilors ---two Senators and two members of the Lower House--- that represented the majority party and the largest minority party in each Chamber.\footnote{In effect, these four seats granted the larger political parties (PRI, PAN, and PRD) direct representation in the Institute's main executive body.}  This incipient scheme of checks and balances was incomplete due to the preponderant presence of the Executive's representative at IFE (the Interior Ministry), who acted as Chairman of the Council General, had the capacity to control agenda items, and could cast a tie-breaking vote in the Council.

IFE's original charter created a second group of six Citizen Councilors to seat in the Council General. Citizen Councilors could in principle reduce political bickering, grant voice and representation to the electorate, and bring into IFE extensive legal and technical know-how (*cite here). The six Citizen Councilors voted in all deliberations and were able to introduce new agenda items as long as these were recognized by the Chairman. Yet, because the first Council General (1994-1996) was still presided by the Interior Ministry, and because large parties were granted votes in all decisions through their Senate and Congressional representatives, questions arose early on concerning the ``autonomy'' of the first Council General.\footnote{To our knowledge, Malo \& Pastor's (1996) account remains the most authoritative analysis of the voting behavior of Councilors from 1994 to 1995. They code information contained in the minutes of all sessions of the Council General between June 1994 and November 1995 and analyze the voting record of six Citizen Councilors and four Legislative Councilors in search of the determinants of the electoral behavior of these individuals. Rosas (2004) inspects the complete voting record of the first IFE Council.}

IFE was reformed in 1996, and this time legislators filled the Council exclusively with Citizen Councilors. Most importantly, the Interior minister lost the faculty to act as Chairman of the Council, which in practice meant that the Executive relinquished day-to-day control over electoral matters  (Brinegar, Morgenstern, \& Nielsen). As we explain below, the Lower Chamber of Congress kept its ability to nominate an enlarged Council General (nine, rather than six, Citizen Councilors). The 1996 reform allegedly turned IFE into an agency autonomous from Executive interference, but as we suggested before, IFE's institutional setup leaves ample room to speculate about potential party-sponsor effects on the voting behavior of Councilors.  We investigate the behavior of this second Council (1996-2003).  We now turn to a detailed discussion of the reformed institutional setup, underscoring first those rules that provide incentives for pro-sponsor behavior, then those that facilitate supra-partisan technocratic consensus.

\subsection{Incentives for partisan or pro-sponsor voting behavior}
As we make clear in this section, IFE's appointment rules lend themselves well to analysis within a standard principal-agent framework.  From this perspective, IFE Councilors are the agents of the enacting coalition in the Lower House.  Parties in the enacting coalition delegate to their appointees authority to interpret the law and run all aspects of federal elections.  The first problem, from the perspective of those in the enacting coalition, becomes how to reduce agency losses that result from the Council General behaving in ways other than those that best serve the principals' interests.  The second problem arises from the fact that the enacting coalition is a collective principal, whose members (political parties) have conflicting interests.\footnote{For a general discussion of the logic of delegation, see Kiewiet and McCubbins (1991: 22-38).}  We emphasize three aspects of this principal-agent situation that are particularly important in producing pro-sponsor behavior: Rules of nomination, signalling devices used by sponsors, and party capture.

\subsubsection{Rules of nomination}
Councilors are appointed by a two-thirds vote in the Lower House of Congress to serve seven-year terms.\footnote{Tenure in office is somewhat secure, since Councilors can only be impeached by a two-thirds vote in the Lower House.}
Legislators have informally agreed in all appointment sessions since 1994 that each party in the enacting coalition is entitled to appoint a share of Councillors roughly proportional to its lower chamber seat share, and that appointee proposals can be vetoed by any other party in the coalition. A final logroll in the lower chamber on a closed list of nine Councillors plus replacements (picked from a list negotiated with the new Citizen Councilors before the start of a term) culminates the process.  In 1996, PRI, PAN, PRD, and PT were in the enacting coalition, leaving no party with Congressional representation behind.\footnote{PRD stands for \emph{Partido de la Revoluci\'on Democr\'atica}, and is the main left-of-center alternative in Mexican politics; PT is the \emph{Partido del Trabajo}. In 2003, PRD and PT were excluded from the enacting coalition, and the \emph{Partido Verde Ecologista Mexicano} was brought in in their stead.} Table \ref{T:proposals} displays information about the enacting coalitions formed in 1996, 2000, and 2003, along with the number of candidates that each party in the coalition saw through the nomination process.

%\ctable[
%   caption = Legislative party shares and Councillor appointments,
%   %width = 100mm
%   label = T:proposals,
%]{lccccc}{
%   \tnote{Enacting coalition in boldface.}
%   \tnote[b]{CG: Council General.}
%}{                                   \FL
%& $57^{\text{th}}$ House\tmark & \multicolumn{2}{$2^{\text{nd}}$ CG\tmark[b]} & $59^{\text{th}}$ House\tmark & $3^{\text{rd}}$ CG \NN
%\cmidrule{3-4}
%Party&  1994-1997  & 1996-2000 & 2000-2003   & 2003-2006     & 2003-2010 \LL
%PAN  & \textbf{119}&2&  2&  \textbf{151}&4 \NN
%PRI  & \textbf{300}&3&  4&  \textbf{224}&4 \NN
%PRD  & \textbf{71} &3&  2&  97          &  \NN
%PT   & \textbf{10} &1&  1&  6           &  \NN
%PVEM &             & &   &  \textbf{17} &1 \NN
%CD   &             & &   &  5           &  \NN
%Total&          500&9&  9&  500         &9 \LL}

\begin{table}
\caption{Legislative party shares and councilor sponsorship (enacting coalition in bold)}\label{T:proposals}
\begin{tabular}{lccccc}
\hline\\ [-1.5ex]
  &  $57^{\text{th}}$ House &  \multicolumn{2}{c}{$2^{\text{nd}}$ Council } &  $59^{\text{th}}$ House & $3^{\text{rd}}$ Council \\
 \cline{3-4}
 Party &  1994-1997 &  1996-2000 &  2000-2003 &  2003-2006 & 2003-2010 \\
\hline\\ [-1ex]
 PAN &  \textbf{119} &  2 &  2 &  \textbf{151} & 4 \\
 PRI &  \textbf{300} &  3 &  4 &  \textbf{224} & 4 \\
 PRD &  \textbf{71} &  3 &  2 &  97 &  \\
 PT &  \textbf{10} &  1 &  1 &  6 &  \\
 PVEM &   &   &   &  \textbf{17} & 1 \\
 CD &   &   &   &  5 &  \\
 Total &  500 &  9 &  9 &  500 & 9 \\
\hline
\end{tabular}
\end{table}

While an informal right to veto may kill off highly partisan (and otherwise unqualified) candidates proposed by other sponsors, it is unlikely that any party would nominate individuals clearly opposed to its own interests and views about electoral regulation.\footnote{In this regard, the dynamic is similar to the one used to fill the US Supreme Court.}
Parties reduce the chances of selecting ``bad types'' ---i.e., individuals that take courses of action hurting the principal's interests--- by carefully screening potential agents.  Parties actively engage in screening, proposing names of people who, though politically unaffiliated, have preferences in line with those of the nominating party. Screening thus helps mitigate agency costs.

Agency costs can also be mitigated through institutional checks.  Here the collective nature of the principal and the inherent conflict of interests among parties, are served well by the collective nature of the agent and the formal and informal rules of appointment.  A stylized view of the nomination process has each party in the enacting coalition choosing types that share its policy objectives, and proposing them to the other coalition members.  Candidates that are too far ideologically from the other coalition members are vetoed, and only moderate names survive.  As in Cox and McCubbins's (1993) congressional committees, we see the resulting Council General as a microcosm of the enacting coalition in the Lower House, with Councilors checking one another as legislative parties would.

\subsubsection{Signalling devices used by sponsors}
Even if Councilors shirk and deviate from their principals' expectations about appropriate voting behavior, parties retain a wide gamut of mechanisms to make their preferences known to agents ---and ultimately to call them to order.  The range includes positioning in legislative commissions and Council session debates, private communications of all sorts, and (in the extreme) threats of impeachment against their own nominees.\footnote{Reference to cases of PRI threats against Merino. Chairman Ugalde was called ``Foxista'' consejera.}by the PRI's Lower-Chamber contingent (who nominated him), after he voiced concerns about the feasibility of new legislation allowing Mexicans abroad to vote; *Reforma, 16 March, 2005.}
It may be very difficult to find evidence for the effectiveness of signalling, however, since it may only lead to the inclusion of a party's concerns in a general agreement with unanimous support.\footnote{Malo \& Pastor (1996) find very mixed evidence in divided votes for the effectiveness of signalling on the basis of legislative councillor votes and authorship of IFE bills.}

\subsubsection{Party capture}
Assuming elected councillors are ambitious and have very low discount rates for the future, their expectations of post-IFE careers may be molded by offers of continued sponsorship in the future (or, indeed, by rival offers from non-sponsors). The possibility of ``party capture'' of Councilors was so obvious that original legislation disallowed them from seeking office after their tenure in IFE.  However, PRI legislators successfully thwarted temporal prohibitions on government jobs and/or electoral candidacies for retired/resigning Councilors. Ironically, Table \ref{T:postife} confirms that the parties that demanded electoral impartiality and citizen control have mostly advanced (rewarded?) the post-IFE careers of their nominees, while the PRI has always abandoned its own. In any case, along with screening and signalling devices, parties can offer post-IFE ``golden parachutes'' to Councilors as an added incentive to engage in pro-sponsor voting behavior.\\

\begin{table}
\caption{Post-IFE Careers of Citizen Councilors}\label{T:postife}
\begin{tabular}{llp{3in}}
\hline
Councilor & Sponsor & Post-IFE career \\ \hline \\ [-1.5ex]
\underline{1994-1996} &   & \\  [0.5ex]
Creel       & PAN   & PAN Congressman, then runner-up to Mexico City Government (PAN), then Secretary of the Interior. \\ [0.5ex]
Woldenberg  & PRI   & President IFE (PRI), then Professor (UNAM). \\ [0.5ex]
Granados    & PRD   & Journalist, then losing PRD candidate to Congress. \\ [0.5ex]
Zertuche    & PAN   & Professor (UNAM), then IFE's Secretary General, then Administrative Tribunal. \\ [0.5ex]
Ortiz       & PRD   & PRD Congressman, then cabinet member of Mexico City's Government. \\ [0.5ex]
Pozas       &       & Professor (UNAM). \\ [1ex]
\underline{1996-2003} & & \\ [0.5ex]
C\'ardenas  & PRD & Professor (UNAM) (resigned blue-ribbon PRD committee). \\ [0.5ex]
Barrag\'an  & PRD & Professor (UNAM).\\ [0.5ex]
Cant\'u     & PRD & PRD's nominee for IFE's Council General, 2003-2010, then Professor (ITESM). \\ [0.5ex]
Zebad\'ua   & PRD & Secretary of the Interior in Chiapas, then PRD Congressman. \\ [0.5ex]
Lujambio    & PAN & Professor (ITAM). \\ [0.5ex]
Molinar     & PAN & Under-secretary of the Interior (Creel), then PAN Congressman. \\ [0.5ex]
Merino      & PRI & Professor (CIDE). \\ [0.5ex]
Peschard    & PRI & Professor (UNAM). \\ [0.5ex]
Luken       & PAN & Currently seeking PAN's gubernatorial candidacy in 2007. \\ [0.5ex]
Rivera      & PRI & Professor (UAZ). \\ \hline
\end{tabular}
\end{table}

\noindent To the extent that these three mechanisms operate, Councilors should be ideologically close, and even sympathetic, to their sponsor. In these circumstances, we entertain two expectations about voting patterns in IFE's General Council. First, we expect IFE voting patterns to dovetail with those observed in the lower chamber. More precisely, since the Council General decides by simple majority, and since no party has ever managed to sponsor five or more Councilors, we should observe coalitions similar to those that form in the Lower House, at least in contested votes. In the Lower House, PRI-PAN rolls and PRI-PRD rolls are most common.
%Guillermo: Is this even remotely true?
%Federico: Here we need a chart that specifies for each Consejo General and each Chamber of Deputies since 1994, the observed incidence of unanimity in voting -- and possibly of quasi-unanimity as well, in order to find parallels for the C�rdenas/Barrag�n anomalies--, as well as the incidence of rolls for the three major parties.
%Guillermo: Can you guys get this?  Jeff probably has it readily available.
Second, a weaker expectation is that same-party appointees should exhibit very similar voting behavior in IFE. Even allowing for slack due to vote-trading and idiosyncratic intensities, we would still expect to find that same-sponsor councillors are closer in behavior to each other---for example, on an ideological scale---than to members sponsored by other parties.  From the perspective of nominating rules, contested votes that do not conform to these two patterns can be considered agency costs.

\subsection{Incentives for supra-partisan behavior}
IFE was trumpeted in its origins as an autonomous EMB that placed major decision-making powers in the hands of non-partisan citizens. We have already referred to the long tenure of its members and the stability of its operative budget as mild guarantees of independence.  By themselves, these features would probably not suffice to induce what we call ``supra-partisan'' behavior. In practice, we see built-in incentives for Councilors to vote together, in a block, as if propelled by an esprit de corps.  This behavior is recognizable in the voting pattern of the $1^{\text{st}}$ Council, where estimated ``ideal points'' of Councilors reflect both a partisan divide and a divide between Citizen and Legislative representatives (see Rosas 2004).  Here, we refer to two major incentives for Councilors to form super-majorities: the threat of impeachment and the existence of a last-appeal electoral tribunal.

%\subsubsection{}
%Guillermo: I'm not sure this is a clear cut argument.  Can we refine it?
%Less important, the other signalling devices (beyond threats of impeachment) would be used to indicate the limits of party tolerance for IFE policy decisions, given the expectations that every party sponsor should have about likely shirking not only by their nominees to the General Council, but also by other parties' nominees.   In other words, these cues also communicate the terms for congressional quiescence.


\subsubsection{Rules of impeachment}
Though the stated objective of the 1996 IFE reform was to grant Citizen Councilors autonomy from parties, the contract retains one important element to constrain their behavior: the threat of impeachment.  Impeachment trials are started by a simple majority in the lower chamber and concluded by a two-thirds vote in the Senate.  In principle, an alliance of any two of the three large parties could have sustained a majority vote against any councilor in the Chamber of Deputies at any moment since 1994, though no alliance could have still passed impeachment in the Senate.%Is this true about the PRI's Senate contingent since 1994?
Under these circumstances, even ideologically-motivated Councilors would shirk in order to protect their flanks against accusations of flagrant partisanship in their behavior.\footnote{Indeed, threats of impeachment have all been characterized by charges of overt partisanship by offending councilors.  The most outrageous example is a recent one, when the PRI and PVEM (with a total of 49\% of Lower House seats) accused Arturo S\'anchez of following the line dictated by his compadre Molinar in San L\'azaro.}%Federico: We need to list the dates and motives of these threats accumulated since 1997.
In order to protect their tenure, Councilors should therefore make sure not to act in ways that systematically hurt the interests of parties with two-thirds support in the lower chamber. Aside from individualized shirking, we would also expect to see cross-party cohesion among councilors on most important issues faced by IFE.  In other words, our expectation for voting behavior based on the incentives of the impeachment process is twofold: Each councilor would try to avoid pure partisan alignment with his or her sponsor and all councilors would seek some minimum of copartisan consensus (unanimity or quasi-unanimity) in their aggregate behavior.
%Eric: El impeachment es por juicio politico que involucra al Senado?

\subsubsection{Potential vetoes by a court of last resort}
Most discussions of IFE's institutional setup tend to omit a second actor, namely, the \emph{ Tribunal Federal Electoral} ({\sc Trife}), created with the 1996 reform.  All Council General decisions are liable to be sent to this federal Electoral Court.  All political parties, national political associations, and even ordinary citizens in some cases, have standing before the {\sc Trife} to challenge or appeal IFE decisions, and indeed the tribunal has over the course of its history shown a steady interest in overturning IFE accords, sometimes rewriting the law and the tribunal's own jurisprudence in order to force its criteria on IFE and other times denying IFE's self-attribution of decision-making power.  In practice, the rulings of the Justices have been remarkably unpredictable, and IFE decisions sent to Court had a good probability of being turned down or amended substantially.%Eric: Story or bit of evidence to support last claim
This unpredictability has spawned litigiousness by those with standing to appeal. Thus, signalling by party sponsors of intentions to appeal should have increased over time.  %Federico: At the very least, we need a time chart indicating the number of sponsor-party appeals to the Tribunal.

More importantly, a Councilor who cared intensely for some resolution had to anticipate all major complaints and make a priori concessions to preempt affected parties from calling on {\sc Trife}.  This could be achieved in two ways.  First, by amending the proposal if necessary, in order to internalize the Court's preferences and avoid a negative ruling.  The unpredictable nature of the Court's behavior has rendered this difficult, if not impossible: {\sc Trife}'s policy preferences remain too fuzzy.  This leaves a second way, namely, reducing the probability that someone will want to object a decision in Court.  This alternative calls for larger, quasi-universal voting coalitions in the Council General. The natural recourse for the councillors, given flightiness from the tribunal and even spurious legal appeals from their sponsors, has been to circle their wagons---that is, to seek safety in broad copartisan consensus.

If this logic is correct, a preliminary summary of the evidence suggests that {\sc Trife} has indeed exerted influence.  As shown in Table \ref{T:valid}, quasi-universal and universal coalitions of nine Councilors voting in the same sense are the norm for the observations that make up our dataset. More dramatically, Figure 1 shows that this very high degree of consensus has been common throughout IFE's history.  The mass of consensual votes (i.e., those where all nine Councilors voted identically, without abstentions) appears sandwiched between the pair of lines.  The slice is quite thick at certain moments, leaving relatively few contested votes (i.e., those where at least one minority vote or one abstention were recorded) throughout the period.\\

\begin{table}
\caption{Valid votes in the $2^{\text{nd}}$ Council General}\label{T:valid}
\begin{tabular}{lcc}
\hline\\ [-1.5ex]
&10/31/96~-&12/11/00~-\\
Dates& 11/14/00& 10/21/03\\ \hline
Citizen Councilors&9&9\\
Total votes&888&491\\
Unanimous votes&640&210\\
Non-unanimous (useful) votes&248&281\\ \hline
\end{tabular}
\end{table}

\noindent Consensus, of course, can simply be the product of similar policy preferences among Councilors.  More to the point, the logic of anticipation detailed above might mean that  Councilors never propose bills that would not command universal support in the Council General, unless they could confidently discount the Court as a non-threat to the winning coalition.  In any case, we now turn to the estimation of ideal points of IFE's Citizen Councilors during the period 1996-2003.  To the extent that pro-sponsor incentives might be dominant, we expect Councilors to line up along an ideological dimension similar to that of legislative parties.  To the extent that supra-partisan incentives are paramount, we expect ideal points that are not clearly distinguished from each other.

\section{Bayesian Estimation of Ideal Points}\label{S:estimation}
At the time of Malo \& Pastor's (1996) analysis of IFE's Council General, political scientists had not yet developed methodological tools to infer the location of bliss points from the voting records of members of small committees.  Since the mid-1990s, however, political methodologists have developed various techniques to circumvent what Londregan (2000) calls ``the micro-committee problem''.  In essence, the micro-committee problem arises from the relative paucity of divided votes that would allow us to infer the ideological positions of committee members.  Among the new techniques, Bayesian estimation methods have recently challenged the dominance of more traditional tools of ideal point estimation (for example, Poole \& Rosenthal's NOMINATE) as the most appropriate ways to study the voting behavior of individuals in small committees (Martin \& Quinn 2002; Clinton, Jackman \& Rivers 2004; Jackman 2004).  Since IFE's Council General is in practice a very small decision body, and since most of the votes in the Council are on procedural---therefore mostly consensual---matters, Bayesian Monte Carlo Markov Chain (MCMC) methods are ideal tools to infer the political preferences of its members.

We present an analysis of IFE's second Council General. We start by noting that halfway through the second Council, one PAN-sponsored Councilor (Molinar) and one PRD-sponsored Councilor (Zebad\'ua) accepted cabinet positions in the Federal and Mexico City governments, and were replaced by Councilors Rivera and Luken, respectively. We estimate ideal points for each of these eleven individuals, but we break down the estimation in two steps, considering nine Councilors each time.  Note also that, as shown in Table~\ref{T:valid}, a large proportion of votes in the Council were unanimous, which in practice means that they convey absolutely no information about the ideological preferences of Councilors.  Only 44\% of votes in the second Council are usable.  Usable votes were recoded so that, in each case, a Councilor's vote with the majority of the Council is coded as ``1'', whereas a minority vote is coded ``0''.  Abstentions are coded as missing values.  The data are thus combined in two arrays of 248 and 281 rows (corresponding to the same number of non-unanimous votes in both halves of the second Council) by 9 columns.

We have included a technical description of the model in Section~\ref{S:model}.  By and large, we follow closely the discussion in Martin \& Quinn (2002) and Clinton, Jackman \& Rivers (2004).  Both sets of authors derive their models from first principles about the voting behavior of individuals in large (the US Lower House) and small (the US Supreme Court) committees.  These models are similar in spirit to those used in item-response theory (IRT), in which individuals' answers to test items of varying difficulty are inspected to infer individual abilities.  The only additional complication in our work is that we estimate ideological positions in a two-dimensional space.  Where appropriate, we explicate our modeling decisions fully, but we note here that in general we stay close to Jackman's discussion (Jackman 2001).

\subsection{One-dimensional models}
We start, however, by estimating ideal points along a single dimension.  As suggested in Section~\ref{S:model}, the identification of IRT models requires imposing restrictions either on item parameters or on Councilors positions.  Traditionally, scholars use a known ``extremist'' in the committee to anchor the ideological space.  We use the alternative method of restricting the prior distributions of ideal points to be standard normal.  Given the relatively large amount of votes (248 and 281), even these mildly informative priors will not impact our substantive results, but allow us to anchor ideal points on a space of known dimensions.

Table~\ref{T:idealpoints} summarizes results from a one-dimensional fit to the data.  The last column in Table~\ref{T:idealpoints} displays the number of votes on which the estimation of ideal points is based for each Councilor.  These are actual YEA/NAY votes; abstentions are not included in this count.  Note that within each Council, point estimates of the ideal positions of Councilors (the mean of the posterior distribution of the 11 location parameters) determine their rank in the list.  Thus, for example, the nine Citizen Councilors that served from 1996 to 2003 are arranged as follows from Left to Right: C\'ardenas, Cant\'u, Zebad\'ua, Lujambio, Molinar, Merino, Woldenberg, Peschard, and Barrag\'an. It is clear from this account that these ideological positions are mildly supportive of the pro-sponsor hypothesis.  The glaring anomaly is Barrag\'an's extreme position to the right of this dimension, when Councilors with the same sponsor (PRD) otherwise occupy the left end of the scale.

\begin{table}
\caption{Posterior distribution of ideal points}\label{T:idealpoints}
\begin{tabular}{llrrr}
\hline
Councilor   &  Sponsor  &   Mean    &  SD & Votes  \\  \hline  \\ [-1ex]

\multicolumn{2}{l}{\underline{1996-2000}}&          &        & \\ [1ex]
C\'ardenas& PRD & --1.623   & 0.1796 & 231\\
Cant\'u   & PRD & --0.114   & 0.0937 & 235\\
Zebad\'ua & PRD & --0.036   & 0.0911 & 238\\
Lujambio  & PAN &   0.364   & 0.0999 & 247\\
Molinar   & PAN &   0.449   & 0.1046 & 234\\
Merino    & PRI &   0.606   & 0.1153 & 247\\
Woldenberg& PRI &   0.619   & 0.1175 & 245\\
Peschard  & PRI &   0.649   & 0.1178 & 247\\
Barrag\'an& PRD &   1.568   & 0.1836 & 206\\ [1ex]
\multicolumn{2}{l}{\underline{2000-2003}}&          &        & \\ [1ex]
C\'ardenas& PRD & --1.323   & 0.1715 & 234\\
Barrag\'an& PRD &   0.141   & 0.0991 & 203\\
Luken     & PAN &   0.888   & 0.1205 & 246\\
Cant\'u   & PRD &   1.081   & 0.1312 & 267\\
Rivera    & PRI &   1.286   & 0.1484 & 271\\
Lujambio  & PAN &   1.381   & 0.1524 & 270\\
Peschard  & PRI &   1.390   & 0.1543 & 270\\
Merino    & PRI &   1.394   & 0.1588 & 276\\
Woldenberg& PRI &   1.479   & 0.1622 & 278\\
\hline
\end{tabular}
\end{table}

It is also noteworthy that the posterior distributions of ideal points in the second Council are wide enough that they overlap in many instances, despite the fact that posterior standard deviations are always much narrower than the prior standard deviation of ``1''.  Thus, for example, the positions of Councilors Merino, Woldenberg, and Peschard are virtually indistinguishable both in 1996-2000 and 2000-2003.  Even then, the voting behavior of the Citizen Councilors is consistent enough that we can venture educated guesses regarding the probable identity of the median voter.\footnote{Since we are reporting probability distributions regarding the location of ideal points, we cannot make deterministic claims about the identity of the median voter.} Sampling from the posterior distribution of ideal points allows us to rank the positions of Councilors. These simulations are summarized in Table~\ref{T:median}.  During the first three years of the second Council, we are very certain that Molinar (sponsored by the PAN) was the median voter.\footnote{Indeed, the probability that he was not the median voter is a paltry 0.026.}  We are less certain about the identity of the median voter during the latter years of the second Council, but even here there is a rather large probability (0.612) that Rivera, who was sponsored by the PRI, reclaimed this status.

\begin{table}
\caption{Probability of being the median voter}\label{T:median}
\begin{tabular}{llcc}
\hline
Councilor & Sponsor &1996-2000&2000-2003\\ \hline
Barrag\'an& PRD     &0.000    & 0.000\\
C\'ardenas& PRD     &0.000    & 0.000\\
Cant\'u   & PRD     &0.000    & 0.012\\
Zebad\'ua & PRD     &0.000    & ---\\
Molinar   & PAN     &0.974    & ---\\
Lujambio  & PAN     &0.022    & 0.132\\
Luken     & PAN     &---      & 0.000\\
Rivera    & PRI     &---      & 0.612\\
Merino    & PRI     &0.000    & 0.116\\
Peschard  & PRI     &0.000    & 0.109\\
Woldenberg& PRI     &0.002    & 0.019\\ \hline
\end{tabular}
\end{table}

%HERE, WE SHOULD MENTION THE "PENTAGONO" FOR THE FIRST TIME
In any case, it is obvious that neither C\'ardenas nor Barrag\'an, two of the most outspoken Councilors sponsored by the PRD, ever had a chance of becoming the median voter in the second Council.  Their ideological positions were simply too extreme to make them dependable as perennial coalition partners.  We argue below that five centrist Councilors, colloquially known as \emph{el Pent\'agono}, consistently banded together to form majority rolls.

\subsection{Two-dimensional models}
We mentioned before that we prefer to estimate a two-dimensional ideological space to the voting data in the second Council.  There are three reasons why we think this is advisable.  First and foremost, our inferences regarding Barrag\'an's ideal point on one dimension suggest an extreme degree of repositioning, even to a larger extent than we had expected. Thus, rather than remaining eccentric vis-\'a-vis other PRD-sponsored Councilors, Barrag\'an's underlying preferences seem more attuned to those of C\'ardenas and Cant\'u during 2000-2003. We had reason to suspect that Barrag\'an sought the good graces of his sponsor precisely in anticipation of the end of his tenure at IFE.  However, the extreme volte-de-face implied by his turn leftwards strikes us as too large to be plausible.  By restricting the ideological space to one dimension, we might be imposing too much structure and thus forcing estimation of weird ideal points.

Second, an advantage of IRT models is that item parameters (i.e., those attached to each bill, case, or vote) can be estimated alongside position parameters.  We have so far reported only on position parameters (Table~\ref{T:idealpoints}), but our inspection of item parameters reveals that only a handful of votes allow discrimination on one dimension.  For the period 1996-2000, for example, we estimate that 87 votes (out of 247) do not convey information that would allow us to discriminate Councilors' ideal points along the inferred dimension.\footnote{To arrive at this conclusion, we estimated the 90\% highest posterior density (HPD) of each parameter.  Where a vote's HPD straddled ``0'', we concluded that it was not informative on that dimension.}  It seems then that our data contain extra information that could allow us to estimate a less parsimonious but more precise model of voting behavior.

Finally, we believe that left-right ideological differences do not extenuate the level of disagreement in the second Council.  On the contrary, our interviews suggest that Councilors were divided on their interpretation of the Council General's legislative functions.  We further believe that these interpretative differences were anchored in the Councilors' different professional backgrounds.  In particular, Councilors with a background in Law (C\'ardenas and Barrag\'an) tended to be punctilious in limiting the Council's legislative faculties, whereas Councilors with a Political Science education tended to be more expansive in their interpretation of the Council's legislative powers, particularly in regard to monitoring party activities.\footnote{We thank Jeff Weldon for offering this insight.}  Consequently, for each period (1996-2000, and 2000-2003) we sought two votes along party lines that would fix the Left-Right dimension that we previously uncovered, and one vote that divided lawyers from political scientists to anchor a second dimension. By stipulating ``spike'' prior distributions on the item parameters of these six votes (three for 1996-2000, three for 2000-2003) we solve specification problems typical of IRT models (see methodological Appendix).  Table~\ref{T:priors} provides details about these six votes.

\begin{table}
\caption{Votes used to anchor two-dimensional models}\label{T:priors}
\begin{tabular}{llp{2.5in}}
\hline
Date   & Minority vote & Substance \\ \hline   \\ [-1ex]
\multicolumn{2}{l}{\underline{1996-2000}}& \\ [1ex]
12/23/1996  & PRD         & Should party finance reform include campaign expenditures?\\[0.5ex]
01/15/1997  & Barrag\'an  & Should \emph{Alianza C\'ivica} be recognized as a National Political Group?\\[0.5ex]
10/14/1999  & PRI         & Should IFE's rules of procedure be amended? \\[1ex]
\multicolumn{2}{l}{\underline{1996-2000}}& \\ [1ex]
04/06/2001  & Barrag\'an  & Should Council pass a resolution against PRI, PAN, and PVEM?\\[0.5ex]
10/21/2001  & PRD         & Should IFE's rules of procedure be amended?\\[0.5ex]
12/21/2001  & PRD, Luken & Should instructions regarding how to obtain National Political Group status be amended?\\ \hline
\end{tabular}
\end{table}

Our results are summarized in Figures~\ref{F:second_session_A} and \ref{F:second_session_B}, which correspond to the first and second halves of IFE's second Council General.  The ellipses in each graph summarize the posterior distributions of the position parameters along two dimensions.  The ellipses are centered at the mean of the distribution, and the lengths along each dimension capture one standard deviation in each direction about the mean.\footnote{Given that the posterior distributions of these parameters are normal, the mean and median should theoretically coincide.  Since we describe the shapes of these distributions with simulated draws, therefore including Montecarlo error in our statistics, the actual means and medians do not exactly coincide. Yet, they are very close to each other and the posterior distributions of these parameters are indeed bell-shaped.}  Smaller ellipses correspond to more certain inferences about the ideological positions of Citizen Councilors.

%This works if translating directly to pdf, but not to dvips-ps
\begin{figure}[t]
  % Requires \usepackage{graphicx}
  \includegraphics[width=130mm]{"C:/Documents and Settings/Guillermo Rosas/My Documents/MY RESEARCH/PROYECTO IFE/paper ERIC/second_session_A"}
  \caption{Ideological dimensions underlying IFE's second Council, 1996-2000}\label{F:second_session_A}
\end{figure}

\begin{figure}[t]
  % Requires \usepackage{graphicx}
  \includegraphics[width=130mm]{"C:/Documents and Settings/Guillermo Rosas/My Documents/MY RESEARCH/PROYECTO IFE/paper ERIC/second_session_B"}
  \caption{Ideological dimensions underlying IFE's second Council, 2000-2003}\label{F:second_session_B}
\end{figure}

Consider first the results in Figure~\ref{F:second_session_A}. The ``ideal regions'' of Councilors Cant\'u and Zebad\'ua (PRD) are estimated with great precision, whereas the voting patterns of Barrag\'an and C\'ardenas only allow a less precise inference.  We do see, however, that the Councilors' positions on dimension 1 remain very similar to those we had uncovered in our one-dimensional model. Here again, we see that the Left-Right positions of the Council are: C\'ardenas, Cant\'u, Zebad\'ua, Lujambio, Molinar, Merino, Woldenberg, Peschard, and Barrag\'an.  The extent of party-sponsor voting is more conspicuous in the case of the PAN---where the positions of Molinar and Lujambio are practically indistinguishable from each other---and the PRI---where Merino, Peschard, and Woldenberg also appear to share similar ideological positions.  We also note that a second dimension, which we here label ``expansion of legislative faculties'', does not really allow much differentiation among most members of the Council.  Indeed, the distribution of position parameters for C\'ardenas along the second dimension is wide enough to cover the inferred positions of all other Councilors except Barrag\'an.  Indeed, this second dimension might only be revealing Barrag\'an's early fixation with keeping to the letter of IFE's charter and avoid trespassing legal limits.

Turning now to Figure~\ref{F:second_session_A}, we see that the ideological space became clearly two-dimensional after the exit of Molinar and Zebad\'ua.  We see here a very conspicuous tendency by the remnants of the \emph{Pent\'agono}---Merino, Woldenberg, Peschard, Lujambio, and Rivera---to vote together, whereas C\'ardenas and Barrag\'an continue to exhibit eccentric ideological positions.  On the second, ideological dimension, C\'ardenas now very clearly overtakes Barrag\'an as the most vocal Councilor against expansion of IFE's law-making faculties.  Note also that Figure~\ref{F:second_session_B} suggests why the one-dimensional fit to the Council's votes in the period 2000-2003 showed such an extreme turnaround for Barrag\'an.  Imposing one dimension to these ideological profiles means setting Councilors' ideal points along an axis that goes from C\'ardenas on the Left to Peschard/Woldenberg on the Right.  The perpendicular projection of Barrag\'an's position onto this axis sets him to the left of Luken, rather than to the right of Peschard/Woldenberg.  We conclude then that Barrag\'an's Left-Right position did not change after the exit of Molinar and Zebad\'ua from the Council.  To the extent that the PRD chooses Councilors expecting them to be faithful to the party line, they really missed the boat when choosing Barrag\'an.

Finally, we report our findings regarding the likely identity of the median voter in each direction, for each of the periods under study, which are summarized in Table~\ref{T:median2d}.  First, during the period 1996-2000 we continue to find that Molinar was the likely median voter along the Left-Right ideological dimension, though the probability that this was indeed the case ($p=0.8$) is now slightly lower than the probability we had estimated in the one-dimensional model ($p=0.974$).  Along the second dimension, our inability to clearly distinguish positions means that we need to consider several candidates as potential median votes.  Among the nine Councilors, Molinar, Lujambio, Merino, Peschard, and Woldenberg, in that order, all have non-negligible probabilities of being the median voter.

\begin{table}
\caption{Probability of being the median voter}\label{T:median2d}
\begin{tabular}{llcccc}
\hline
Councilor & Sponsor &\multicolumn{2}{c}{1996-2000}&\multicolumn{2}{c}{2000-2003}\\ \hline
Barrag\'an& PRD     &0.039&0.009& 0.001&0.000\\
C\'ardenas& PRD     &0.000&0.011& 0.001&0.000\\
Cant\'u   & PRD     &0.002&0.046& 0.137&0.004\\
Zebad\'ua & PRD     &0.011&0.070& ---  &---  \\
Molinar   & PAN     &0.801&0.243& ---  &---  \\
Lujambio  & PAN     &0.073&0.208& 0.185&0.077\\
Luken     & PAN     &---  &---  & 0.001&0.041\\
Rivera    & PRI     &---  &---  & 0.222&0.750\\
Merino    & PRI     &0.051&0.159& 0.183&0.049\\
Peschard  & PRI     &0.007&0.141& 0.136&0.057\\
Woldenberg& PRI     &0.018&0.114& 0.135&0.031\\ \hline
\end{tabular}
\end{table}

This situation is reversed for the second half of the Council, where we have a likely median voter on the ``expansion of legislative faculties'' dimension (Rivera, with $p=0.75$ of enjoying median voter status), but not on the ideological Left-Right dimension (Rivera, Lujambio, Merino, Cant\'u, Peschard, and Woldenberg, in that order, could have been the median voter).  In any case, it seems obvious that PAN-sponsored Councilors lost the median position to PRI-sponsored Councilors halfway through the second Council.

\section{Conclusion}\label{S:discussion}
The purpose of this research note was to describe the spatial location of Citizen and Legislative Councilors in IFE's Council General from 1994 to 2003.   I did so within a Bayesian framework, using MCMC techniques to describe and analyze the posterior distribution of ideal points.  In principle, this analysis can and should be extended in several directions.

First, I chose to anchor Councilor Woldenberg's position as a matter of mere convenience.  It would be more interesting to develop informative priors that codify the perceived ideological positions of IFE Councilors.  For example, given the very high abstention rates of Senator Porfirio Mu\~noz Ledo during the first Council, and the fact that he was a representative of a leftist party, it would be convenient to use him as an anchor on the negative side of the spectrum (that is, the region usually associated with the Left).  For the second Council, it might be more productive to anchor Councilor C\'ardenas on the Left, though my impression is that his position was so eccentric that this expedient would only ``flip'' the results presented in Table~\ref{T:idealpoints}.

More interestingly, one could estimate ideal positions in spaces of higher dimensionality.  In practice, two dimensions should be more than enough to accommodate other potential issues that might reflect ideological cleavages among Councilors.  In this case, it is impractical and undesirable to constrain the positions of Councilors, as defining a two-dimensional space would require fixing at least three such positions---already a third of the Council.  Instead, the usual practice is to define a two-dimensional space by choosing bills or issues that might have cleaved the Council in recognizable ways, and using them to fix axes of competition (for example, an economic axis and a cultural axis orthogonal to the first).

Finally, the analysis could very profitably be extended to the voting behavior of electoral authorities in other countries (or even in subnational units within Mexico).  Of particular relevance is the issue of the relative autonomy and objectivity of members of these bodies.  Returning to Molina \& Hern\'andez's typology, would we expect to find more proficient yet accountable electoral authorities in ``party watchdog'' systems or in ``technocratic'' institutional setups?  What trade-offs are posed by alternative ways of organizing the electoral authority?  Will the process of \emph{ciudadanizaci\'on} of the Mexican electoral authority bring ideological conformity as the price of technical proficiency, or will this be achieved without sacrificing the representation of multiple points of view?  In short, is IFE the best electoral authority that taxpayers' money can buy?

\section{The Model}\label{S:model}
The voting behavior of individuals in small committees conveys information about their policy preferences.  Whether these preferences are sincerely revealed during the voting process or whether they reflect some contrived strategic calculus is subject of debate, but not a point that requires further discussion in the context of this paper.\footnote{In any case, decisions in the Council General are reached by majority voting on binary outcomes---even if the bill under consideration is ``modified'' during the discussion prior to a vote.  One could argue that preferences revealed by IFE's Councilors are likely to be sincere, rather than strategic, because there are no incentives for strategic misrepresentation of preferences in an up-or-down vote.  Obviously, a Councilor can still abstain from voting, which introduces an element of strategy in an otherwise straightforward voting decision.}
Sincere or strategic motivations apart, it is incumbent upon the researcher to specify the mechanism that presumably links political preferences to vote choices.  Though it is not the only modeling option, most political scientists rely on the Euclidean spatial model to build up their analysis from solid first principles (Ordeshook 1976, Hinich \& Munger 1994).  Put succinctly, spatial models assume that, when facing a binary YEA or NAY vote choice, rational committee members will vote for the alternative that will enact the policy closest to their own ideal position.    I follow Martin \& Quinn (2002) and Clinton, Jackman \& Rivers (2004) in formalizing this utility calculation as follows:
Let $U_{i}(\boldsymbol{\zeta}_{j})= - \norm{\mathbf{x}_{i}-\boldsymbol{\zeta}_{j}}^{2}+\eta_{i,j}$ represent the utility to committee member $i \in I_{n}$ of voting in favor of proposal $j \in J_{m}$ and $U_{i}(\boldsymbol{\psi}_{j})= -\norm{\mathbf{x}_{i}-\boldsymbol{\psi}_{j}}^{2} + \nu_{i,j}$ the utility of voting against it.
In this formalization, the $D$-dimensional vectors $\mathbf{x}_{i}$, $\boldsymbol{\zeta}_{j}$, and $\boldsymbol{\psi}_{j}$ correspond, respectively, to the ideal position of the committee member in the policy space, the position that will result from a YEA vote, and the position that will result from a NAY vote.  In many empirical applications---and in this research note---ideal points are estimated in one-dimensional space, though nothing in this model's formulation precludes estimation of ideal points in multidimensional spaces.\footnote{Poole \& Rosenthal (2001) note that one and at most two ideological dimensions generally suffice to capture the policy preferences of members in large committees.}
The disturbances $\eta_{i,j}$ and $\nu_{i,j}$ are assumed to be distributed joint-normally with zero means and known variance (again, these assumptions can be relaxed to accommodate other error structures).

To turn this formal utility notation into a statistical model susceptible of estimation, note that a positive vote by member $i$ on proposal $j$ ($y_{i,j}=1$) reveals that $U_{i}(\zeta_{j})$ $ \geq  U_{i}(\psi_{j})$ (though, because of the stochastic components $\eta_{i,j}$  and $\nu_{i,j}$, it is not necessarily true that $\norm{\mathbf{x}_{i}-\boldsymbol{\zeta}_{j}} \leq \norm{\mathbf{x}_{i}-\boldsymbol{\psi}_{j}}$).  Conversely, a negative vote by member $i$ on proposal $j$ ($y_{i,j}=0$) suggests that $U_{i}(\zeta_{j})$ $ \leq  U_{i}(\psi_{j})$.  From these relations, it follows that a committee member will decide to vote YEA on any given proposal if $U_{i}(\zeta_{j})$$- U_{i}(\psi_{j}) > 0$:

\begin{align}\label{E:equation1}
y_{i,j}
   &= U_{i}(\zeta_{j} ) - U_{i}(\psi_{j}) \\
   &=  -\norm{x_{i}-\zeta_{j}}^{2}+\eta_{i,j} +\norm{x_{i}-\psi_{j}}^{2}+\nu_{i,j} \nonumber \\
   &=  2(\eta_{j}-\psi_{j})x_{i} + \psi_{j}^{2}-\zeta_{j}^{2}+ \eta_{i,j} +\nu_{i,j} \nonumber \\
   &= \alpha_{j} + \beta_{j} x_{i} + \varepsilon_{i,j}, \nonumber
\end{align}

\noindent where $\alpha_{j}=\psi_{j}^{2}-\zeta_{j}^{2}$,  $\beta_{j}= 2(\eta_{j}-\psi_{j})$, and $\varepsilon_{i,j}=\eta_{i,j} +\nu_{i,j}$.  The last line in Equation~(\ref{E:equation1}) can be rearranged to represent each vote $y_{i,j}$ as an independent draw from a normal probability distribution; thus $p(y_{i,j}=1) = \int_{0}^{\infty} \Phi(\alpha_{j}+\beta_{j} x_{i})$, where $\Phi(\cdot)$ is the normal cumulative distribution function.   If, for notational convenience, the parameters $\alpha_{j}$, $\beta_{j}$, and $x_{i}$ are stacked in vectors $\boldsymbol{\alpha}$, $\boldsymbol{\beta}$, and $\mathbf{x}$ (of lengths $m$, $m$, and $n$ respectively), the likelihood function can be constructed from the observed $\mathbf{Y}$:

\begin{equation}\label{E:equation2}
\mathcal{L}(\boldsymbol{\alpha},\boldsymbol{\beta},\mathbf{x}|\mathbf{y}) = \prod_{j=1}^{m} \prod_{i=1}^{n}  \Phi(\alpha_{j}+\beta_{j} x_{i})^{y_{i,j}} (1-\Phi(\alpha_{j}+\beta_{j} x_{i}))^{1-y_{i,j}}
\end{equation}

The likelihood function in Equation~(\ref{E:equation2}) can be estimated statistically.  Note however that we require estimates of $\boldsymbol{\alpha}$ and $\boldsymbol{\beta}$ (the case parameters), and $\mathbf{x}$ (the ideal points of councilors, the only parameters of relevance for the purposes of this paper), and that we only have information collected in the matrix $\mathbf{Y}$ of observed votes (0's and 1's) for all committee members on all proposals discussed by IFE's Council General.  As it stands, thus, the model is not identified, because an infinite number of values of $\boldsymbol{\alpha}$, $\boldsymbol{\beta}$, and $\mathbf{x}$ are solutions to the system of $j$ equations in (\ref{E:equation1}).\footnote{There are two sources of under-identification in item response models: scale invariance and rotational invariance.  See the discussion in Martin \& Quinn (2002: 139) and Clinton, Jackman \& Rivers (2004: 356-357).  Note also that, in the context of Bayesian estimation, proper priors on the $\boldsymbol{\alpha}$, $\boldsymbol{\beta}$, and $\mathbf{x}$ parameters help solve the identification problem.}
Thus, in order to allow identification of the model parameters, it is necessary to add restrictions on their possible values.  For example, factor analysis and principal component analysis restrict correlation among different dimensions to be orthogonal.  In the methods deviced by Martin \& Quinn (2002) and Clinton, Jackman \& Rivers (2004), one can alternatively fix $\mathbf{x}_{i}$ for ``known'' holders of extreme views in the committee, or fix $\boldsymbol{\alpha}$ and $\boldsymbol{\beta}$ parameters for some bills or decisions.  (See the discussion in Section~\ref{S:estimation}).

Since the approach pursued here is Bayesian, rather than frequentist, I treat observed votes by IFE's councilors as fixed quantities, and the $\boldsymbol{\alpha}$, $\boldsymbol{\beta}$, and $\mathbf{x}$ parameters as random variables from a joint posterior distribution in need of estimation.  In the Bayesian approach, prior information about these parameters can be combined with the likelihood function (specified in (\ref{E:equation2})) to obtain posterior distributions of the parameters of interest.  I start with similar priors for the parameters $\boldsymbol{\alpha}$, $\boldsymbol{\beta}$, and $\mathbf{x}$ in the model:

\begin{align}\label{E:equation3}
p(\boldsymbol{\alpha}) &\sim \mathcal{N}_{J}(\mathbf{0},\mathbf{1})\\
p(\boldsymbol{\beta}) &\sim \mathcal{N}_{J}(\mathbf{0},\mathbf{1})\nonumber \\
p(\mathbf{x})  &\sim \mathcal{N}_{I}(\mathbf{0},\mathbf{1}) \nonumber
\end{align}

The sole exception, required for identification purposes, is a constraint on the spatial position of one Councilor (see Section~\ref{S:estimation}). The joint posterior distribution of $\boldsymbol{\alpha}$, $\boldsymbol{\beta}$, and $\mathbf{x}$ results from the product of the likelihood function in (\ref{E:equation2}) and the set of prior distributions in (\ref{E:equation3}), as expressed in (\ref{E:equation4}):

\begin{equation}\label{E:equation4}
\pi(\boldsymbol{\alpha}, \boldsymbol{\beta}, \mathbf{x}|\mathbf{y}) \propto \mathcal{L}(\boldsymbol{\alpha},\boldsymbol{\beta},\mathbf{x}|\mathbf{y}) p(\boldsymbol{\alpha})p(\boldsymbol{\beta})p(\mathbf{x})
\end{equation}

I estimate the posterior distribution in Equation~(\ref{E:equation4}) through  MCMC routines using Martin \& Quinn's (2002) \emph{MCMCpack} for \texttt{R}.

%The following code sets up PoliSci-looking bibliographic items%
\section*{References}
\mbox{} \baselineskip=6pt \parskip=1.1\baselineskip plus 4pt minus 4pt \vspace{-\parskip}

\bibitem Clinton, Joshua, Simon Jackman, and Douglas Rivers. 2004. ``The Statistical Analysis of Roll Call Data''. \emph{American Political Science Review}, 98 (2), May, 355-370.

\bibitem Hinich, Marvin, and Michael C. Munger. 1994. \emph{Ideology and the theory of public choice}. Ann Arbor: University of Michigan Press.

\bibitem Londregan, John. 2000. \emph{Legislative Institutions and Ideology in Chile's Democratic Transition}.  New York: Cambridge University Press.

\bibitem Martin, Andrew D., and Kevin M. Quinn. 2002. ``Dynamic Ideal Point Estimation via Markov Chain Monte Carlo for the U.S. Supreme Court, 1953-1999''. \emph{Political Analysis}, 10 (2), Spring, 134-153.

\bibitem Molina, Jos\'e, and Janeth Hern\'andez. 1999. ``La credibilidad de las elecciones latinoamericanas y sus factores.  el efecto de los organismos electorales, el sistema de partidos y las actitudes pol\'iticas''. \emph{Cuadernos del Cendes}, 41, mayo-agosto, 1-26.

\bibitem Ordeshook, Peter C. 1976. ``The spatial theory of elections: A review and a critique''. In I. Budge, I. Crewe, \& D. Farlie (eds.), \emph{Party identification and beyond}.  London: John Wiley \& Sons.

\bibitem Pastor, Julio, and Ver\'onica Malo. 1996. \emph{Autonom\'ia e imparcialidad en el Consejo General del IFE, 1994-1995}. M\'exico: Instituto Tecnol\'ogico Aut\'onomo de M\'exico.  Unpublished senior's thesis.

\bibitem Poole, Keith T., and Howard Rosenthal. 1997. \emph{Congress: A Political-Economic History of Roll Call Voting}. New York: Oxford University Press.

\bibitem Poole, Keith T., and Howard Rosenthal. 2001. ``D-NOMINATE after 10 years: A comparative update to \emph{Congress: A political-economic history of roll-call voting}''. \emph{Legislative Studies Quarterly}, 26 (1), 5-29.



\end{document}
